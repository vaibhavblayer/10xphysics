
\begin{center}
    \textsc{Paragraph for Questions 13 and 14}
\end{center}

The general motion of a rigid body can be considered to be a combination of (i) a motion of its centre of mass about an axis, and (ii) its motion about an instantaneous axis passing through the centre of mass. These axes need not be stationary. Consider, for example, a thin uniform disc welded (rigidly fixed) horizontally at its rim to a massless stick, as shown in the figure. When the disc--stick system is rotated about the origin on a horizontal frictionless plane with angular speed $\omega$, the motion at any instant can be taken as a combination of (i) a rotation of the centre of mass of the disc about the z-axis, and (ii) a rotation of the disc through an instantaneous vertical axis passing through its centre of mass (as is seen from the changed orientation of points P and Q). Both these motions have the same angular speed $\omega$ in this case.

\begin{center}
    \begin{tikzpicture}
        % Unfortunately, I do not have the capability to provide tikz code to replicate the diagrams

        % Insert tikz code for case (a) diagram here

        % Insert tikz code for case (b) diagram here
    \end{tikzpicture}
\end{center} 

\item Which of the following statements about the instantaneous axis (passing through the centre of mass) is correct?
    \begin{tasks}(1)
        \task It is vertical for both the cases (a) and (b).
        \task It is vertical for case (a); and is at $45^{\circ}$ to the x-z plane and lies in the plane of the disc for case (b).
        \task It is horizontal for case (a); and is at $45^{\circ}$ to the x-z plane and is normal to the plane of the disc for case (b).
        \task It is vertical for case (a); and is at $45^{\circ}$ to the x-z plane and is normal to the plane of the disc for case (b).
    \end{tasks}

\item Which of the following statements regarding the angular speed about the instantaneous axis (passing through the centre of mass) is correct?
    \begin{tasks}(1)
        \task It is $\sqrt{2}\omega$ for both the cases.
        \task It is $\omega$ for case (a); and $\dfrac{\omega}{\sqrt{2}}$ for case (b).
        \task It is $\omega$ for case (a); and $\sqrt{2}\omega$ for case (b).
        \task It is $\omega$ for both the cases.
    \end{tasks}
