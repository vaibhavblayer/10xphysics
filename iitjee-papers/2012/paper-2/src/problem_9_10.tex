

\begin{center}
    \textsc{Paragraph for Questions 9 and 10}
\end{center}

Most materials have the refractive index, \( n > 1 \). So, when a light ray from air enters a naturally occurring material, then by Snell's law, \(\frac{\sin \theta_1}{\sin \theta_2} = \frac{n_2}{n_1}\), it is understood that the refracted ray bends towards the normal. But it never emerges on the same side of the normal as the incident ray. According to electromagnetism, the refractive index of the medium is given by the relation, \(n = \left( \frac{c}{v} \right) = \sqrt{\epsilon_r \mu_r}\), where \(c\) is the speed of electromagnetic waves in vacuum, \(v\) its speed in the medium, \(\epsilon_r\) the relative permittivity and \(\mu_r\) the permeability of the medium respectively.

In normal materials, both \(\epsilon_r\) and \(\mu_r\) are positive, implying positive \(n\) for the medium. When both \(\epsilon_r\) and \(\mu_r\) are negative, one must choose the negative root of \(n\). Such negative refractive index materials can now be artificially prepared and are called meta-materials. They exhibit significantly different optical behavior, without violating any physical laws. Since \(n\) is negative, it results in a change in the direction of propagation of the refracted light. However, similar to normal materials, the frequency of light remains unchanged upon refraction even in meta-materials.

\begin{center}
    \begin{tikzpicture}
        % As I cannot replicate the exact diagrams from the image, I'll draw a simple representation
        % Diagram A
        \draw (-4,0) -- (0,0); % Surface line
        \draw[-stealth] (-3,-1) -- (-2,0); % Incident ray
        \draw[dashed] (-2,0) -- (-2,1); % Normal line
        \draw[-stealth] (-2,0) -- (-1,0.5); % Refracted ray diagram A
        
        % Diagram B (similar to A, mirrored vertically)
        \draw (0,0) -- (4,0); % Surface line
        \draw[-stealth] (1,-1) -- (2,0); % Incident ray
        \draw[dashed] (2,0) -- (2,1); % Normal line
        \draw[-stealth] (2,0) -- (3,-0.5); % Refracted ray diagram B
        
        \node at (-2,-1.5) {(A)};
        \node at (2,-1.5) {(B)};
        
        % As the actual diagrams also include (C) and (D), extend as needed
        % Nodes can represent the diagrams' labels
        
    \end{tikzpicture}
\end{center} 

\item For light incident from air on a meta-material, the appropriate ray diagram is
    \begin{tasks}(2)
        \task Diagram A
        \task Diagram B
        \task Diagram C
        \task Diagram D
    \end{tasks} 

\item Choose the correct statement.
    \begin{tasks}(1)
        \task The speed of light in the meta-material is \( v = \frac{c}{|n|} \)
        \task The speed of light in the meta-material is \( v = \frac{c}{n} \)
        \task The speed of light in the meta-material is \( v = c \)
        \task The wavelength of the light in the meta-material \( \lambda_m \) is given by \( \lambda_m = \lambda_{\text{air}}|n|\), where \( \lambda_{\text{air}} \) is the wavelength of the light in air.
    \end{tasks} 
