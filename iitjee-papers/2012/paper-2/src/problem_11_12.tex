\begin{center}
    \textsc{Paragraph for Questions 11 and 12}
\end{center}

The $\beta$-decay process, discovered around 1900, is basically the decay of a neutron $(n)$. In the laboratory, a proton $(p)$ and an electron $(e^-)$ are observed as the decay products of the neutron. Therefore, considering the decay of a neutron as a two-body decay process, it was predicted theoretically that the kinetic energy of the electron should be a constant. But experimentally, it was observed that the electron kinetic energy has a continuous spectrum. Considering a three-body decay process, i.e. $n \rightarrow p + e^- + \bar{\nu}_e$, around 1930, Pauli explained the observed electron energy spectrum. Assuming the anti-neutrino $(\bar{\nu}_e)$ to be massless and possessing negligible energy, and the neutron to be at rest, momentum and energy conservation principles are applied. From this calculation, the maximum kinetic energy of the electron is $0.8 \times 10^6$ eV. The kinetic energy carried by the proton is only the recoil energy.

\item What is the maximum energy of the anti-neutrino?
    \begin{tasks}(2)
        \task Zero.
        \task Much less than $0.8 \times 10^6$ eV.
        \task Nearly $0.8 \times 10^6$ eV.
        \task Much larger than $0.8 \times 10^6$ eV.
    \end{tasks}

\item If the anti-neutrino had a mass of $3 eV/c^2$ (where $c$ is the speed of light) instead of zero mass, what should be the range of the kinetic energy, $K$, of the electron?
    \begin{tasks}(2)
        \task $0 \leq K \leq 0.8 \times 10^6$ eV
        \task $3.0 eV \leq K \leq 0.8 \times 10^6$ eV
        \task $3.0 eV \leq K < 0.8 \times 10^6$ eV
        \task $0 \leq K < 0.8 \times 10^6$ eV
    \end{tasks}