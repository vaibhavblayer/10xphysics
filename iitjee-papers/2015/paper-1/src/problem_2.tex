

    \item Consider a hydrogen atom with its electron in the \(n^{th}\) orbital. An electromagnetic radiation of wavelength 90 nm is used to ionize the atom. If the kinetic energy of the ejected electron is 10.4 eV, then the value of \(n\) is (\(hc = 1242\) eV nm)
    \underline{\hspace{2.5 cm}}

    \begin{solution}
        \begin{align*}
            \intertext{First calculate the energy of the incident electromagnetic radiation.}
            E &= \dfrac{hc}{\lambda}\\
            &= \dfrac{1242 \,\text{eV nm}}{90 \,\text{nm}}\\
            &= 13.8 \,\text{eV}\\
            \intertext{The kinetic energy of the ejected electron is given as \(10.4 \,\text{eV}\). Hence, the energy used to ionize the atom is the difference.}
            E_{\text{ionization}} &= E - KE\\
            &= 13.8 \,\text{eV} - 10.4 \,\text{eV}\\
            &= 3.4 \,\text{eV}\\
            \intertext{For hydrogen atom, the energy of \(n^{th}\) orbital is given by \(E_n = - \dfrac{13.6\, \text{eV}}{n^2}\). This means the energy required to ionize from \(n^{th}\) orbital is \(E_n\). So,}
            E_{\text{ionization}} &= -E_n\\
            \dfrac{13.6\, \text{eV}}{n^2} &= 3.4\, \text{eV}\\
            n^2 &= \dfrac{13.6\, \text{eV}}{3.4\, \text{eV}}\\
            n^2 &= 4\\
            n &= \sqrt{4}\\
            n &= 2\\
            \intertext{Therefore, the value of \(n\) is \(2\).}
        \end{align*}
    \end{solution}

