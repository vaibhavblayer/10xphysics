
\item A nuclear power plant supplying electrical power to a village uses a radioactive material of half life \( T \) years as the fuel. The amount of fuel at the beginning is such that the total power requirement of the village is \( 12.5\% \) of the electrical power available from the plant at that time. If the plant is able to meet the total power needs of the village for a maximum period of \( nT \) years, then the value of \( n \) is \underline{\hspace{2.5 cm}}.

\begin{solution}
    \begin{align*}
        \intertext{Let the initial amount of radioactive material be $A_0$ and the initial power output be $P_0$.}
        \text{Power requirement of the village} &= 0.125 \times P_0\\
        \intertext{After $nT$ years, the amount of radioactive material remaining will be $A_0 \left(\frac{1}{2}\right)^n$.}
        \intertext{The power output at this time will be proportional to the amount of radioactive material remaining.}
        \text{Power output after $nT$ years} &= P_0 \left(\frac{1}{2}\right)^n\\
        \intertext{Since this power output must meet the power requirement of the village, we have}
        P_0 \left(\frac{1}{2}\right)^n &= 0.125 \times P_0\\
        \left(\frac{1}{2}\right)^n &= 0.125\\
        2^{-n} &= \frac{1}{8}\\
        -n &= \log_2 \frac{1}{8}\\
        -n &= -3\\
        n &= 3
        \intertext{Therefore, the value of $n$ is 3.}
    \end{align*}  
\end{solution}
