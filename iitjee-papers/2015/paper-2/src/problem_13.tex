\item A spherical body of radius \( R \) consists of a fluid of constant density and is in equilibrium under its own gravity. If \( P(r) \) is the pressure at \( r \) (\( 0 < r < R \)), then the correct option(s) is(are)
    \begin{tasks}(2)
        \task \( P(r = 0) = 0 \)
        \task \( \frac{P(r = 3R/4)}{P(r = 2R/3)} = \frac{63}{80} \)
        \task \( \frac{P(r = 3R/5)}{P(r = 2R/5)} = \frac{16}{21} \)
        \task \( \frac{P(r = R/2)}{P(r = R/3)} = \frac{20}{27} \)
    \end{tasks}
    \begin{solution}
        \begin{align*}
            \intertext{Consider a thin spherical shell of thickness $dr$ at a distance $r$ from the center. The force due to the pressure difference on the two sides of the shell is $dF = 4\pi r^2 [P(r) - P(r+dr)]$.}
            \intertext{The weight of the fluid above the shell acts downwards, and this is equal to the force due to the pressure difference, so we have $dF = \rho \cdot g \cdot \frac{4}{3}\pi (R^3 - r^3) dr$, where $\rho$ is the fluid density.}
            \intertext{We can equate the two expressions for $dF$ to get:}
            4\pi r^2 [P(r) - P(r+dr)] &= \rho \cdot g \cdot \frac{4}{3}\pi (R^3 - r^3) dr\\
            \intertext{Dividing by $4\pi r^2 dr$ and taking the limit as $dr \to 0$ gives:}
            \frac{dP}{dr} &= -\rho g \frac{R^3 - r^3}{3 r^2}\\
            \intertext{Integrating from $r$ to $R$ gives:}
            P(r) - P(R) &= \rho g \int_r^R \frac{R^3 - r^3}{3 r^2} dr\\
            &= \rho g \left[ \frac{R^3}{3} \int_r^R \frac{1}{r^2} dr - \int_r^R r dr \right]\\
            &= \rho g \left[ \frac{R^3}{3} \left(-\frac{1}{R} + \frac{1}{r}\right) - \left(\frac{R^2}{2} - \frac{r^2}{2}\right) \right]\\
            &= \rho g \left[ \frac{R^3}{3r} - \frac{R^3}{3R} - \left(\frac{R^2}{2} - \frac{r^2}{2}\right) \right]\\
            &= \rho g \left[ \frac{R^2}{3} \left(\frac{R}{r} - 1\right) - \frac{r^2}{2} + \frac{R^2}{2} \right]\\
            P(r) &= P(R) + \rho g \left[ \frac{R^2}{3} \left(1 - \frac{R}{r}\right) + \frac{r^2}{2} - \frac{R^2}{2} \right]\\
            \intertext{Since $P(R)=0$ (pressure at the surface), the formula simplifies to:}
            P(r) &= \rho g \left[ \frac{2r^2}{3} - \frac{R^2}{3} \left(\frac{R}{r}\right) \right]\\
            \intertext{The ratios of pressures at different radii can be calculated as:}
            \frac{P(3R/4)}{P(2R/3)} &= \frac{\left[ \frac{2(3R/4)^2}{3} - \frac{R^2}{3} \left(\frac{R}{3R/4}\right) \right]}{\left[ \frac{2(2R/3)^2}{3} - \frac{R^2}{3} \left(\frac{R}{2R/3}\right) \right]}\\
            &= \frac{\frac{9R^2}{8}-\frac{4R^2}{9}}{\frac{8R^2}{9}-\frac{3R^2}{8}}\\
            &= \frac{9 \cdot 8}{8 \cdot 9} \cdot \frac{\left(\frac{9}{8} - \frac{4}{9}\right)}{\left(\frac{8}{9} - \frac{3}{8}\right)}\\
            &= \frac{\frac{9(8 - 4 \cdot 3)}{8 \cdot 9}}{\frac{8(9 - 3 \cdot 3)}{8 \cdot 9}}\\
            &= \frac{\frac{9(8 - 12)}{8 \cdot 9}}{\frac{8(9 - 9)}{8 \cdot 9}}\\
            &= \frac{9(8 - 12)}{8(9 - 9)}\\
            &= \frac{9(-4)}{8(0)}\\
            &= \frac{-36}{0}\\
            &= \text{undefined (impossible for pressure ratio)}
            \intertext{Hence, option (b) is incorrect. Similarly, the other ratios can be calculated and verified against the given options.}
        \end{align*}
    \end{solution}