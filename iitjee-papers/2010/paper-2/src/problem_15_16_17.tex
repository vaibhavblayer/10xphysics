
\begin{center}
    \textsc{Paragraph for Questions 53 to 55}
\end{center}

The key feature of Bohr's theory of spectrum of hydrogen atom is the quantization of angular momentum when an electron is revolving around a proton. We will extend this to a general rotational motion to find quantized rotational energy of a diatomic molecule assuming it to be rigid. The rule to be applied is Bohr's quantization condition.

\item A diatomic molecule has moment of inertia \( I \). By Bohr's quantization condition its rotational energy in the \( n^{th} \) level \( (n = 0 \) is not allowed) is
    \begin{tasks}(2)
        \task \(\dfrac{1}{n^2}\left(\dfrac{h^2}{8\pi^2 I}\right)\)
        \task \(\dfrac{1}{n}\left(\dfrac{h^2}{8\pi^2 I}\right)\)
        \task \(n\left(\dfrac{h^2}{8\pi^2 I}\right)\)
        \task \(n^2\left(\dfrac{h^2}{8\pi^2 I}\right)\)\ans
    \end{tasks}

\item It is found that the excitation frequency from ground to the first excited state of rotation for the CO molecule is close to \( \dfrac{4}{\pi} \times 10^{11} \) Hz. Then the moment of inertia of CO molecule about its center of mass is close to \( \left(\text{Take } h = 2 \times 10^{-34} J s\right) \)
    \begin{tasks}(2)
        \task \(2.76 \times 10^{-46} kg m^2\)
        \task \(1.87 \times 10^{-46} kg m^2\)\ans
        \task \(4.67 \times 10^{-47} kg m^2\)
        \task \(1.17 \times 10^{-47} kg m^2\)
    \end{tasks}

\item In a CO molecule, the distance between C (mass = 12 a.m.u.) and O (mass = 16 a.m.u.), where 1 a.m.u. \( = \dfrac{5}{3} \times 10^{-27} kg \), is close to
    \begin{tasks}(4)
        \task \(2.4 \times 10^{-10} m\)
        \task \(1.9 \times 10^{-10} m\)
        \task \(1.3 \times 10^{-10} m\)\ans
        \task \(4.4 \times 10^{-11} m\)
    \end{tasks}
