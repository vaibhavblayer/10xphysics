    \item A human body has a surface area of approximately 1 \( m^2 \). The normal body temperature is 10 K above the surrounding room temperature \( T_0 \). Take the room temperature to be \( T_0 = 300 \) K. For \( T_0 = 300 \) K, the value of \( \sigma T^4 = 460 \) Wm\(^{-2}\) (where \( \sigma \) is the Stefan-Boltzmann constant). Which of the following options is/are correct?
        \begin{tasks}(1)
            \task The amount of energy radiated by the body in 1 second is close to 60 Joules
            \task If the surrounding temperature reduces by a small amount \( \Delta T \ll T_0 \), then to maintain the same body temperature the same (living) human being needs to radiate \( \Delta W = 4 \sigma T_0^3 \Delta T \) more energy per unit time
            \task Reducing the exposed surface area of the body (e.g. by curling up) allows humans to maintain the same body temperature while reducing the energy lost by radiation
            \task If the body temperature rises significantly then the peak in the spectrum of electromagnetic radiation emitted by the body would shift to longer wavelengths
        \end{tasks}