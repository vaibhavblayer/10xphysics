

    \item A uniform circular disc of mass $50 \, \text{kg}$ and radius $0.4 \, \text{m}$ is rotating with an angular velocity of $10 \, \text{rad s}^{-1}$ about its own axis, which is vertical. Two uniform circular rings, each of mass $6.25 \, \text{kg}$ and radius $0.2 \, \text{m}$, are gently placed symmetrically on the disc in such a manner that they are touching each other along the axis of the disc and are horizontal. Assume that the friction is large enough such that the rings are at rest relative to the disc and the system rotates about the original axis. The new angular velocity $\underline{\hspace{2.5 cm}}$ (in $\text{rad s}^{-1}$) of the system is

