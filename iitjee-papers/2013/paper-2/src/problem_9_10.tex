\begin{center}
    \textsc{Paragraph for Questions 9 and 10}
\end{center}

A point charge $Q$ is moving in a circular orbit of radius $R$ in the x-y plane with an angular velocity $\omega$. This can be considered as equivalent to a loop carrying a steady current $\frac{Q\omega}{2\pi}$. 

A uniform magnetic field along the positive z-axis is now switched on, which increases at a constant rate from $0$ to $B$ in one second. Assume that the radius of the orbit remains constant. The application of the magnetic field induces an emf in the orbit. The induced emf is defined as the work done by an induced electric field in moving a unit positive charge around a closed loop. It is known that, for an orbiting charge, the magnetic dipole moment is proportional to the angular momentum with a proportionality constant $\gamma$. 

\item The magnitude of the induced electric field in the orbit at any instant of time during the time interval of the magnetic field change is
    \begin{tasks}(4)
        \task $\dfrac{BR}{4}$
        \task $\dfrac{BR}{2}$\ans
        \task $\dfrac{BR}{\pi}$
        \task $2BR$
    \end{tasks}

\item The change in the magnetic dipole moment associated with the orbit, at the end of the time interval of the magnetic field change, is
    \begin{tasks}(4)
        \task $-\gamma BQR^2$
        \task $-\dfrac{\gamma BQR^2}{2}$\ans
        \task $\dfrac{BQR^2}{\gamma}$
        \task $\gamma BQR^2$
    \end{tasks}