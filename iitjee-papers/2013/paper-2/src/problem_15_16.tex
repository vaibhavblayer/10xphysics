
\begin{center}
    \textsc{Paragraph for Questions 15 and 16}
\end{center}

A thermal power plant produces electric power of 600 \emph{kW} at 4000 \emph{V}, which is to be transported to a place 20 \emph{km} away from the power plant for consumers' usage. It can be transported either directly with a cable of large current carrying capacity or by using a combination of step-up and step-down transformers at the two ends. The drawback of the direct transmission is the large energy dissipation. In the method using transformers, the dissipation is much smaller. In this method, a step-up transformer is used at the plant side so that the current is reduced to a smaller value. At the consumers' end, a step-down transformer is used to supply power to the consumers at the specified lower voltage. It is reasonable to assume that the power cable is purely resistive and the transformers are ideal with a power factor unity. All the currents and voltages mentioned are rms values.

\item If the direct transmission method with a cable of resistance 0.4 $\Omega$ \emph{km}$^{-1}$ is used, the power dissipation (in \%) during transmission is
    \begin{tasks}(4)
        \task 20
        \task 30 \ans
        \task 40
        \task 50
    \end{tasks}

\item In the method using the transformers, assume that the ratio of the number of turns in the primary to that in the secondary in the step-up transformer is 1 : 10. If the power to the consumers has to be supplied at 200 \emph{V}, the ratio of the number of turns in the primary to that in the secondary in the step-down transformer is
    \begin{tasks}(4)
        \task 200 : 1 \ans
        \task 150 : 1
        \task 100 : 1
        \task 50 : 1
    \end{tasks}
