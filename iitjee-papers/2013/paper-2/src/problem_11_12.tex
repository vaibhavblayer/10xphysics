\begin{center}
    \textsc{Paragraph for Questions 11 and 12}
\end{center}

The mass of a nucleus $^{A}_{Z}\mathrm{X}$ is less than the sum of the masses of (A-Z) number of neutrons and Z number of protons in the nucleus. The energy equivalent to the corresponding mass difference is known as the binding energy of the nucleus. A heavy nucleus of mass M can break into two light nuclei of masses $m_1$ and $m_2$ only if $(m_1+m_2) < M$. Also two light nuclei of masses $m_3$ and $m_4$ can undergo complete fusion and form a heavy nucleus of mass M' only if $(m_3+m_4) > M'$. The masses of some neutral atoms are given in the table below:

% Table omitted - table environments are not handled by simple LaTeX environments and require more complex structuring.

\item The correct statement is
    \begin{tasks}(1)
        \task The nucleus $^{6}_{3}\mathrm{Li}$ can emit an alpha particle.
        \task The nucleus $^{210}_{84}\mathrm{Po}$ can emit a proton.
        \task Deuteron and alpha particle can undergo complete fusion.
        \task The nuclei $^{70}_{30}\mathrm{Zn}$ and $^{84}_{34}\mathrm{Se}$ can undergo complete fusion.
    \end{tasks}
    

\item The kinetic energy (in \textit{keV}) of the alpha particle, when the nucleus $^{210}_{84}\mathrm{Po}$ at rest undergoes alpha decay, is
    \begin{tasks}(4)
        \task 5319
        \task 5422
        \task 5707
        \task 5818
    \end{tasks}
    