\begin{center}
    \textsc{Paragraph for Questions 13 and 14}
\end{center}

A small block of mass 1 kg is released from rest at the top of a rough track. The track is a circular arc of radius 40 m. The block slides along the track without toppling and a frictional force acts on it in the direction opposite to the instantaneous velocity. The work done in overcoming the friction up to the point \( Q \), as shown in the figure below, is 150 J. (Take the acceleration due to gravity, \( g = 10\ m/s^2 \)).

\begin{center}
    \begin{tikzpicture}
        % Track
        \draw[thick] (0,0) arc (180:240:2cm);
        % Block sliding marks
        \draw[dashed] (0,0) -- (1.73, -1); % Extended line that fades
        % Radius lines
        \draw[dashed] (0,0) -- (-2,0) node[midway, above] {\( R \)};
        \draw[dashed] (1.73, -1) -- (0,0) node[midway, above right] {\( R \)};
        % Angle
        \draw (0,-0.5) arc (270:300:0.5cm) node[midway, right] {\( 30^\circ \)};
        % Point labels
        \filldraw[black] (0,0) circle (2pt) node[above left] {\( P \)};
        \filldraw[black] (1.73, -1) circle (2pt) node[below right] {\( Q \)};
        % Axis labels
        \draw[->] (-2.5,0) -- (2.5,0) node[right] {\( x \)};
        \draw[->] (0,0) -- (0,2.5) node[above] {\( y \)};
    \end{tikzpicture}
\end{center}

\item The speed of the block when it reaches the point \( Q \) is
    \begin{tasks}(4)
        \task \( 5\ m/s^{-1} \)
        \task \( 10\ m/s^{-1} \)
        \task \( 10\sqrt{3}\ m/s^{-1} \)
        \task \( 20\ m/s^{-1} \)
    \end{tasks}
    
\item The magnitude of the normal reaction that acts on the block at the point \( Q \) is
    \begin{tasks}(4)
        \task \( 7.5\ N \)
        \task \( 8.6\ N \)
        \task \( 11.5\ N \)
        \task \( 22.5\ N \)
    \end{tasks}