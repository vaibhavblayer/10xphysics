
\begin{enumerate}
    \item A free hydrogen atom after absorbing a photon of wavelength \(\lambda_a\) gets excited from the state \(n = 1\) to the state \(n = 4\). Immediately after that the electron jumps to \(n = m\) state by emitting a photon of wavelength \(\lambda_e\). Let the change in momentum of the atom due to the absorption and the emission are \(\Delta p_a\) and \(\Delta p_e\), respectively. If \(\lambda_a/\lambda_e = \frac{1}{5}\), which of the option(s) is/are correct? \\ [Use \(hc = 1242 \ eV\ nm\); \(1 nm = 10^{-9}\) m; \(h\) and \(c\) are Planck's constant and speed of light, respectively.]
        \begin{tasks}(2)
            \task \(m = 2\)
            \task \(\lambda_e = 418 nm\)
            \task \(\Delta p_a/\Delta p_e = \frac{1}{2}\)
            \task The ratio of kinetic energy of the electron in the state \(n = m\) to the state \(n = 1\) is \(\frac{1}{4}\)
        \end{tasks}
\end{enumerate}
