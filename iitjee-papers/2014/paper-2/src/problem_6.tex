
\item Parallel rays of light of intensity \( I = 912\ \text{Wm}^{-2} \) are incident on a spherical black body kept in surroundings of temperature \( 300\ \text{K} \). Take Stefan-Boltzmann constant \( \sigma = 5.7 \times 10^{-8}\ \text{Wm}^{-2}\text{K}^{-4} \) and assume that the energy exchange with the surroundings is only through radiation. The final steady state temperature of the black body is close to
    \begin{tasks}(2)
        \task \( 330\ \text{K} \)\ans
        \task \( 660\ \text{K} \)
        \task \( 990\ \text{K} \)
        \task \( 1550\ \text{K} \)
    \end{tasks}

    \begin{solution}
        \begin{align*}
            \intertext{At steady state, the power absorbed by the black body is equal to the power radiated.}
            \intertext{Power absorbed by the black body per unit area is equal to the intensity of the incident rays, \( I \).}
            \intertext{Power radiated by the black body per unit area is given by Stefan-Boltzmann law: }
            P_{\text{radiated}} &= \sigma T^4
            \intertext{where \( \sigma \) is the Stefan-Boltzmann constant and \( T \) is the temperature of the black body.}
            \intertext{At steady state,}
            I &= \sigma T^4 - \sigma T_s^4
            \intertext{Given that the surroundings are at temperature \( T_s = 300\ \text{K} \), the equation becomes:}
            912 &= 5.7 \times 10^{-8} (T^4 - 300^4)\\
            \intertext{Solving for \( T^4 \):}
            T^4 - 300^4 &= \frac{912}{5.7 \times 10^{-8}}\\
            T^4 &= 300^4 + \frac{912}{5.7 \times 10^{-8}}\\
            T^4 &= 300^4 + 1.6 \times 10^{10}\\
            \intertext{Since \( 300^4 \) is significantly smaller compared to \( 1.6 \times 10^{10} \), we can approximate:}
            T^4 &\approx 1.6 \times 10^{10}\\
            T &\approx \left(1.6 \times 10^{10}\right)^{1/4}\\
            T &\approx 989.9\ \text{K}
            \intertext{Thus, the final steady state temperature of the black body is closest to 990 K.}
            \intertext{Option (c) is correct.}
        \end{align*}
    \end{solution}
