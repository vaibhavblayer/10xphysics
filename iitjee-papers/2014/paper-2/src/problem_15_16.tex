
\begin{center}
    \textsc{Paragraph For Questions 15 \& 16}
\end{center}

The figure shows a circular loop of radius $a$ with two long parallel wires (numbered 1 and 2) all in the plane of the paper. The distance of each wire from the centre of the loop is $d$. The loop and the wires are carrying the same current $I$. The current in the loop is in the counterclockwise direction if seen from above.

\begin{center}
    \begin{tikzpicture}
        % Extra diagram elements (not provided in the reference code)
        % which may include the circular loop, wires, and annotations
        % should be drawn here as per the figure from the question.
    \end{tikzpicture}
\end{center} 

\item When $d > a$ but wires are not touching the loop, it is found that the net magnetic field on the axis of the loop is zero at a height $h$ above the loop. In that case
    \begin{tasks}(1)
        \task current in wire 1 and wire 2 is the direction PQ and RS, respectively, and $h \approx a$
        \task current in wire 1 and wire 2 is the direction PQ and SR, respectively, and $h \approx a$
        \task current in wire 1 and wire 2 is the direction PQ and SR, respectively, and $h \approx 1.2a$
        \task current in wire 1 and wire 2 is the direction PQ and RS, respectively, and $h \approx 1.2a$
    \end{tasks}

\item Consider $d > a$, and the loop is rotated about its diameter parallel to the wires by $30^\circ$ from the position shown in the figure. If the currents in the wires are in the opposite directions, the torque on the loop at its new position will be (assume that the net field due to the wires is constant over the loop)
    \begin{tasks}(2)
        \task $\dfrac{\mu_0 I^2 a^2}{d}$
        \task $\dfrac{\mu_0 I^2 a^2}{2d}$
        \task $\dfrac{3\sqrt{3} \mu_0 I^2 a^2}{d}$
        \task $\dfrac{3\sqrt{3} \mu_0 I^2 a^2}{2d}$
    \end{tasks}
