
\item A student is performing an experiment using a resonance column and a tuning fork of frequency 244s$^{-1}$. He is told that the air in the tube has been replaced by another gas (assume that the column remains filled with the gas). If the minimum height at which resonance occurs is $(0.350 \pm 0.005)$ m, the gas in the tube is

(Useful information: $\sqrt{167RT} = 640 \, J^{1/2}\,mol^{-1/2}$, $\sqrt{140RT} = 590 \, J^{1/2}\,mol^{-1/2}$. The molar masses $M$ in grams are given in the options. Take the values of $\frac{10}{\sqrt{M}}$ for each gas as given there.)

\begin{tasks}(2)
\task Neon $(M = 20, \frac{10}{\sqrt{20}} = \frac{7}{10})$
\task Nitrogen $(M = 28, \frac{10}{\sqrt{28}} = \frac{3}{5})$
\task Oxygen $(M = 32, \frac{10}{\sqrt{32}} = \frac{9}{16})$
\task Argon $(M = 36, \frac{10}{\sqrt{36}} = \frac{17}{32})$
\end{tasks}
