
\item A light source, which emits two wavelengths \( \lambda_1 = 400 \text{ nm} \) and \( \lambda_2 = 600 \text{ nm} \), is used in a Young’s double slit experiment. If recorded fringe widths for \( \lambda_1 \) and \( \lambda_2 \) are \( \beta_1 \) and \( \beta_2 \) and \( \text{C} \) number of fringes for them within a distance \( y \) on one side of the central maximum are \( m_1 \) and \( m_2 \), respectively, then
    \begin{tasks}(2)
        \task \( \beta_2 > \beta_1 \)
        \task \( m_1 > m_2 \)
        \task From the central maximum, \( 3^{rd} \) maximum of \( \lambda_2 \) overlaps with \( 5^{th} \) minimum of \( \lambda_1 \)
        \task The angular separation of fringes for \( \lambda_1 \) is greater than \( \lambda_2 \)
    \end{tasks}
