

    \item During Searle's experiment, zero of the Vernier scale lies between $3.20 \times 10^{-2}\, \mathrm{m}$ and $3.25 \times 10^{-2}\, \mathrm{m}$ of the main scale. The $20^\text{th}$ division of the Vernier scale exactly coincides with one of the main scale divisions. When an additional load of $2\, \mathrm{kg}$ is applied to the wire, the zero of the Vernier scale still lies between $3.20 \times 10^{-2}\, \mathrm{m}$ and $3.25 \times 10^{-2}\, \mathrm{m}$ of the main scale but now the $45^\text{th}$ division of Vernier scale coincides with one of the main scale divisions. The length of the thin metallic wire is $2\, \mathrm{m}$ and its cross-sectional area is $8 \times 10^{-7}\, \mathrm{m}^2$. The least count of the Vernier scale is $1.0 \times 10^{-5}\, \mathrm{m}$. The maximum percentage error in the Young's modulus of the wire is \underline{\hspace{2.5 cm}}

