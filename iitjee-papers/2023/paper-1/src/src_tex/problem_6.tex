\item One mole of an ideal gas expands adiabatically from an initial state \((T_A, V_0)\) to final state \((T_F, 5V_0)\). Another mole of the same gas expands isothermally from a different initial state \((T_B, V_0)\) to the same final state \((T_F, 5V_0)\). The ratio of the specific heats at constant pressure and constant volume of this ideal gas is \(\gamma\). What is the ratio \(T_A/T_B\)?
        \begin{tasks}(2)
            \task \(5^{\gamma-1}\)\ans
            \task \(5^{1-\gamma}\)
            \task \(5^\gamma\)
            \task \(5^{1+\gamma}\)
        \end{tasks}

\begin{solution}
    \begin{align*}
        \intertext{For adiabatic process,}
        \left(\dfrac{T_A}{T_F}\right)^{1-\gamma} &= \dfrac{V_F}{V_A}\\
        \intertext{For isothermal process,}
        T_B &= T_F\\
        \intertext{Given that \(V_F = 5V_0, V_A=V_0\),}
        \dfrac{T_A}{T_F} &= 5^{\gamma-1}\\
        \Rightarrow \dfrac{T_A}{T_B} &= 5^{\gamma-1}
        \intertext{Therefore, the correct option is (a).}
    \end{align*}
\end{solution}