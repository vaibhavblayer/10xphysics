\item A Hydrogen-like atom has atomic number Z. Photons emitted in the electronic transitions from level \( n = 4 \) to level \( n = 3 \) in these atoms are used to perform photoelectric effect experiment on a target metal. The maximum kinetic energy of the photoelectrons generated is 1.95 eV. If the photoelectric threshold wavelength for the target metal is 310 nm, the value of Z is \underline{\hspace{5em}}. \ansint{3}

    [\textit{Given: \( hc = 1240 \) eV-nm and \( Rhc = 13.6 \) eV, where \( R \) is the Rydberg constant, \( h \) is the Planck's constant and \( c \) is the speed of light in vacuum}]

\begin{solution}
    \begin{align*}
        \intertext{The energy difference between the two levels is given by:}
        \Delta E &= E_4 - E_3\\
        &= -\dfrac{Z^2Rhc}{16} + \dfrac{Z^2Rhc}{9}\\
        &= Z^2Rhc\left(\frac{1}{9}-\frac{1}{16}\right)\\
        &= Z^2Rhc\left(\frac{7}{144}\right)\\
        \intertext{The energy of the photon is equal to the energy difference between the two levels:}
        \frac{hc}{\lambda} + \text{K.E.}_{\text{max}} &= \Delta E\\
        \frac{1240}{310} + 1.95 &= Z^2 \cdot 13.6 \cdot \frac{7}{144}\\
        4 + 1.95 &= Z^2 \cdot 0.66\\
        5.95 &= Z^2 \cdot 0.66\\
        Z^2 &= \frac{5.95}{0.66}\\
        Z^2 &= 9\\
        \Rightarrow Z &= 3
        \intertext{Therefore, the value of Z is 3.}
    \end{align*}
\end{solution}