
\begin{enumerate}
    \item A train with cross-sectional area $S_t$ is moving with speed $v_t$ inside a long tunnel of cross-sectional area $S_0$ ($S_0 = 4S_t$). Assume that almost all the air (density $\rho$) in front of the train flows back between its sides and the walls of the tunnel. Also, the air flow with respect to the train is steady and laminar. Take the ambient pressure and that inside the train to be $p_0$. If the pressure in the region between the sides of the train and the tunnel walls is $p$, then $p_0 - p = \frac{7}{2N}\rho v_t^2$. The value of $N$ is \underline{\hspace{2cm}}.
\end{enumerate}
