
\begin{enumerate}
    \item A hot air balloon is carrying some passengers, and a few sandbags of mass 1 kg each so that its total mass is 480 kg. Its effective volume giving the balloon its buoyancy is \( V \). The balloon is floating at an equilibrium height of 100 m. When \( N \) number of sandbags are thrown out, the balloon rises to a new equilibrium height close to 150 m with its volume \( V \) remaining unchanged. If the variation of the density of air with height from the ground is \( \rho(h) = \rho_0 e^{-\frac{h}{h_0}} \), where \( \rho_0 = 1.25 \text{ kg m}^{-3} \) and \( h_0 = 6000 \text{ m} \), the value of \( N \) is \_\_\_\_\_.
\end{enumerate}
