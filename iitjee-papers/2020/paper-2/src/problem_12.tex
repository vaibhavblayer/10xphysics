
\begin{enumerate}
    \item Starting at time \( t = 0 \) from the origin with speed \( 1 \, \text{ms}^{-1} \), a particle follows a two-dimensional trajectory in the \( x-y \) plane so that its coordinates are related by the equation \( y = \frac{x^2}{2} \). The \( x \) and \( y \) components of its acceleration are denoted by \( a_x \) and \( a_y \), respectively. Then
        \begin{tasks}(2)
            \task \( a_x = 1 \, \text{ms}^{-2} \) implies that when the particle is at the origin, \( a_y = 1 \, \text{ms}^{-2} \)
            \task \( a_x = 0 \) implies \( a_y = 1 \, \text{ms}^{-2} \) at all times
            \task at \( t = 0 \), the particle’s velocity points in the \( x \)-direction
            \task \( a_x = 0 \) implies that at \( t = 1 \, s \), the angle between the particle’s velocity and the \( x \) axis is \( 45^\circ \)
        \end{tasks}
\end{enumerate}
