
\begin{enumerate}
    \item A container with 1 kg of water in it is kept in sunlight, which causes the water to get warmer than the surroundings. The average energy per unit time per unit area received due to the sunlight is 700 Wm$^{-2}$ and it is absorbed by the water over an effective area of 0.05 m$^2$. Assuming that the heat loss from the water to the surroundings is governed by Newton’s law of cooling, the difference (in $^\circ$C) in the temperature of water and the surroundings after a long time will be \underline{\hspace{3cm}}. (Ignore effect of the container, and take constant for Newton’s law of cooling = 0.001 s$^{-1}$, Heat capacity of water = 4200 J kg$^{-1}$ K$^{-1}$)
\end{enumerate}
