 1. An open-ended U-tube of uniform cross-sectional area contains water (density \(10^3\text{ kg m}^{-3}\)). Initially, the water level stands at \(0.29\text{ m}\) from the bottom in each arm. Kerosene (a water-immiscible liquid) of density \(800\text{ kg m}^{-3}\) is added to the left arm until its length is \(0.1\text{ m}\), as shown in the schematic figure below. The ratio \(\frac{h_1}{h_2}\) of the heights of the liquid in the two arms is
$$\begin{tasks}(2)
\task \frac{15}{14}\\
\task \frac{35}{33}\\
\task \frac{7}{6}\\
\task \frac{5}{4}
\end{tasks}$$