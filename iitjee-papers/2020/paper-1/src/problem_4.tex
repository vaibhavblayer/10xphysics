 4. A circular coil of radius \(R\) and \(N\) turns has negligible resistance. As shown in the schematic figure, its two ends are connected to two wires and it is hanging by those wires with its plane being vertical. The wires are connected to a capacitor with charge \(Q\) through a switch. The coil is in a horizontal uniform magnetic field \(B_0\) parallel to the plane of the coil. When the switch is closed, the capacitor gets discharged through the coil in a very short time. By the time the capacitor is discharged fully, magnitude of the angular momentum gained by the coil will be (assume that the discharge time is so short that the coil has hardly rotated during this time)
$$\begin{tasks}(2)
\A 2\pi NQB_0R^2 \\\
\B \pi NQB_0R^2 \\\
\C 2\pi N^2QB_0R^2 \\\
\D 4\pi N^2QB_0R^2
\end{tasks}$$