 8. The filament of a light bulb has surface area \(64\text{ mm}^2\). The filament can be considered as a black body at temperature \(2500\text{ K}\) emitting radiation like a point source when viewed from far. At night the light bulb is observed from a distance of \(100\text{ m}\). Assume the pupil of the eyes of the observer to be circular with radius \(3\text{ mm}\). Then 

(Take Stefan-Boltzmann constant \(=5.67\times10^{-8}\text{ W m}^{-2}\text{K}^{-4}\), Wien's displacement constant \(=2.90\times10^{-3}\text{ m K}\), Planck's constant \(=6.63\times10^{-34}\text{ Js}\), speed of light in vacuum \(=3.00\times10^8\text{ ms}^{-1}\))

(A) power radiated by the filament is in the range \(642\text{ W} \) to \(645\text{ W}\)

(B) radiated power entering into one eye of the observer is in the range \(3.15\times10^{-8}\text{ W}\) to \(3.25\times10^{-8}\text{ W}\)

(C) the wavelength corresponding to the maximum intensity of light is \(1160\text{ nm}\)

(D) taking the average wavelength of emitted radiation to be \(1740\text{ nm}\), the total number of photons entering per second into one eye of the observer is in the range \(2.75\times10^{11}\) to \(2.85\times10^{11}\)