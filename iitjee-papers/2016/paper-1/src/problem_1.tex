
\begin{enumerate}
    \item In a historical experiment to determine Planck's constant, a metal surface was irradiated with light of different wavelengths. The emitted photoelectron energies were measured by applying a stopping potential. The relevant data for the wavelength (\(\lambda\)) of incident light and the corresponding stopping potential (\(V_0\)) are given below:
    \begin{center}
        \begin{tabular}{ccc}
        \hline
        \(\lambda (\mu m)\) & \(V_0 (Volt)\) \\
        \hline
        0.3 & 2.0 \\
        0.4 & 1.0 \\
        0.5 & 0.4 \\
        \hline
        \end{tabular}
    \end{center}
    Given that \( c = 3 \times 10^8 m\ s^{-1} \) and \( e = 1.6 \times 10^{-19} C \), Planck's constant (in units of J s) found from such an experiment is
    \begin{tasks}(2)
        \task \( 6.0 \times 10^{-34} \)
        \task \( 6.4 \times 10^{-34} \)
        \task \( 6.6 \times 10^{-34} \)
        \task \( 6.8 \times 10^{-34} \)
    \end{tasks}
\end{enumerate}
