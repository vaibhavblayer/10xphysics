\item Column I shows four systems, each of the same length \( L \), for producing standing waves. The lowest possible natural frequency of a system is called its fundamental frequency, whose wavelength is denoted as \( \lambda_f \). Match each system with statements given in Column II describing the nature and wavelength of the standing waves. 

\begin{center}
    \begin{tikzpicture}
        % Drawing the diagrams for each system
        % Diagram A
        \draw (0,0) -- (4,0);
        \fill[pattern=north east lines] (4,-0.1) rectangle (4.2,0.1);

        % Diagram B
        \draw (0,-1) -- (4,-1);
        \fill[pattern=north east lines] (0,-1.1) rectangle (-0.2,-0.9);
        \fill[pattern=north east lines] (4,-1.1) rectangle (4.2,-0.9);

        % Diagram C
        \draw (0,-2) -- (4,-2);
        \fill[pattern=north east lines] (0,-2.1) rectangle (0.2,-1.9);
        \fill[pattern=north east lines] (4,-2.1) rectangle (4.2,-1.9);

        % Diagram D
        \draw (0,-3) -- (4,-3);
        \fill[pattern=north east lines] (0,-3.1) rectangle (0.2,-2.9);
        \fill[pattern=north east lines] (4,-3.1) rectangle (4.2,-2.9);
        \draw (2,-3) -- (2,-2.6);

        % Adding labels for (o) and (L)
        \foreach \y in {0, -1, -2, -3}
        {
            \draw (-0.2, \y + 0.2) -- (-0.2, \y - 0.2) node[left] {\( 0 \)};
            \draw (4.2, \y + 0.2) -- (4.2, \y - 0.2) node[right] {\( L \)};
        }

        % Adding midpoint label (L/2) for Diagram D
        \draw (2,-3.4) -- (2,-3.8) node[below] {\( L/2 \)};

        % Adding labels A, B, C, D
        \node[left] at (-1,0) {(A)};
        \node[left] at (-1,-1) {(B)};
        \node[left] at (-1,-2) {(C)};
        \node[left] at (-1,-3) {(D)};

        % Adding labels for type of wave (p, q, r, s, t)
        \node[right] at (5,0) {(p)};
        \node[right] at (5,-1) {(q)};
        \node[right] at (5,-2) {(r)};
        \node[right] at (5,-3) {(s)};
        \node[right] at (5,-4) {(t)};
    \end{tikzpicture}
\end{center}

\begin{center}
    \renewcommand{\arraystretch}{2}
    \begin{table}[h]
        \centering
        \begin{tabular}{p{0.25cm}p{8cm}|p{0.25cm}p{5cm}}
        \hline
        & Column I & &Column II \\
        \hline
        (A)& Pipe closed at one end & (p) &Longitudinal waves\\
        (B)& Pipe open at both ends & (q) &Transverse waves\\
        (C)& Stretched wire clamped at both ends & (r) &\( \lambda_f = L \)\\
        (D)& Stretched wire clamped at both ends and at mid-point & (s) &\( \lambda_f = 2L \)\\
        & & (t) &\( \lambda_f = 4L \)\\
        \hline
        \end{tabular}
    \end{table}
\end{center}

\begin{tasks}(2)
    \task \( A \rightarrow p \) and \( t \)
    \task \( B \rightarrow p \) and \( s \)
    \task \( C \rightarrow q \) and \( s \)
    \task \( D \rightarrow q \) and \( r \)
\end{tasks}