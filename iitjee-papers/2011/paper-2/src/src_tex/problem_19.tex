
\item One mole of a monatomic ideal gas is taken through a cycle ABCDA as shown in the P-V diagram. Column I gives the processes involved in the cycle. Match them with the characteristics in Column II.

\begin{center}
    \begin{tikzpicture}
    % Drawing the P-V diagram
    \draw[->] (0,0) -- (0,5) node[anchor=south] {P};
    \draw[->] (0,0) -- (10,0) node[anchor=west] {V};
    \draw (0,1) -- (1,1) -- (1,3) -- (3,1) -- (9,1) -- cycle;
    
    % Points A, B, C, D in the diagram
    \draw (1,3) node[anchor=south east] {B};
    \draw (3,1) node[anchor=north west] {C};
    \draw (9,1) node[anchor=west] {D};
    \draw (0,1) node[anchor=east] {A};
    
    % Dashed lines to axes
    \draw[dashed] (1,3) -- (1,0) node[anchor=north] {1V};
    \draw[dashed] (3,1) -- (0,1) node[anchor=east] {1P};
    \draw[dashed] (3,3) -- (3,0) node[anchor=north] {3V};
    \draw[dashed] (9,1) -- (0,1) node[anchor=east] {3P};
    
    \end{tikzpicture}
\end{center}

\begin{center}
    \renewcommand{\arraystretch}{1.5}
    \begin{table}[h]
        \centering
        \begin{tabular}{p{0.25cm}p{8cm}|p{0.25cm}p{5cm}}
        \hline
        & Column I & &Column II \\
        \hline
        (A)& Process A $\rightarrow$ B & (p) &Internal energy decreases.\\
        (B)& Process B $\rightarrow$ C & (q) &Internal energy increases.\\
        (C)& Process C $\rightarrow$ D & (r) &Heat is lost.\\
        (D)& Process D $\rightarrow$ A & (s) &Heat is gained.\\
        && (t) &Work is done on the gas.\\
        \hline
        \end{tabular}
    \end{table}
\end{center}

\begin{tasks}(2)
    \task A: p, r and t
    \task B: p and r
    \task C: q and s
    \task D: r and t
\end{tasks}
