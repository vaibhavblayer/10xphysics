\begin{center}
    \textsc{Comprehension for Question Nos. 38 and 39}
\end{center}

A dense collection of equal number of electrons and positive ions is called neutral plasma. Certain solids containing fixed positive ions surrounded by free electrons can be treated as neutral plasma. Let 'N' be the number density of free electrons, each of mass 'm'. When the electrons are subjected to an electric field, they are displaced relatively away from the heavy positive ions. If the electric field becomes zero, the electrons begin to oscillate about the positive ions with a natural angular frequency '$\omega_p$', which is called the plasma frequency. To sustain the oscillations, a time varying electric field needs to be applied that has an angular frequency $\omega$, where a part of the energy is absorbed and a part of it is reflected. As $\omega$ approaches $\omega_p$, all the free electrons are set to resonance together and all the energy is reflected. This is the explanation of high reflectivity of metals.

\item Taking the electronic charge as 'e' and the permittivity as '$\epsilon_0$', use dimensional analysis to determine the correct expression for $\omega_p$.
    \begin{tasks}(2)
        \task $\dfrac{Ne}{\sqrt{m\epsilon_0}}$
        \task $\sqrt{\dfrac{m\epsilon_0}{Ne}}$
        \task $\sqrt{\dfrac{Ne^2}{m\epsilon_0}}$\ans
        \task $\sqrt{\dfrac{m\epsilon_0}{Ne^2}}$
    \end{tasks}

\item Estimate the wavelength at which plasma reflection will occur for a metal having the density of electrons N $\approx 4 \times 10^{27} m^{-3}$. Take $\epsilon_0 \approx 10^{-11}$ and m $\approx 10^{-30}$, where these quantities are in proper SI units.
    \begin{tasks}(2)
        \task 800 nm
        \task 600 nm\ans
        \task 300 nm
        \task 200 nm
    \end{tasks}