\item A moving coil galvanometer has 50 turns and each turn has an area \(2 \times 10^{-4}\) m\(^2\). The magnetic field produced by the magnet inside the galvanometer is 0.02 T. The torsional constant of the suspension wire is \(10^{-4} N m/rad^{-1}\). When a current flows through the galvanometer, a full scale deflection occurs if the coil rotates by 0.2 rad. The resistance of the coil of the galvanometer is 50 \(\Omega\). This galvanometer is to be converted into an ammeter capable of measuring current in the range 0 – 1.0 A. For this purpose, a shunt resistance is to be added in parallel to the galvanometer. The value of this shunt resistance, in ohms, is \underline{\hspace{3cm}}.
