\item A particle of mass \( m \) is initially at rest at the origin. It is subjected to a force and starts moving along the x-axis. Its kinetic energy \( K \) changes with time as \( \frac{dK}{dt} = \gamma t \), where \( \gamma \) is a positive constant of appropriate dimensions. Which of the following statements is (are) true?
    \begin{tasks}(1)
        \task The force applied on the particle is constant
        \task The speed of the particle is proportional to time
        \task The distance of the particle from the origin increases linearly with time
        \task The force is conservative
    \end{tasks}
    \begin{solution}
        \begin{align*}
            \intertext{Let's start with the definition of kinetic energy:}
            K &= \frac{1}{2}mv^2\\
            \intertext{According to the question:}
            \frac{dK}{dt} &= \gamma t\\
            \intertext{From the definition of kinetic energy:}
            \frac{dK}{dt} &= mv\frac{dv}{dt}\\
            \intertext{Comparing with the given equation:}
            mv\frac{dv}{dt} &= \gamma t\\
            \intertext{Integrating both sides with respect to time:}
            \int_0^v mv dv &= \int_0^t \gamma t dt\\
            \frac{1}{2}mv^2 &= \frac{1}{2}\gamma t^2 \\
            v^2 &= \frac{\gamma}{m} t^2
            \intertext{As Kinetic energy is always increasing, the velocity of the particle is always positive. Therefore,}
            \Rightarrow v &= \sqrt{\frac{\gamma}{m}} t\\
            \intertext{Now, we can get force using velocity:}
            F &= m\frac{dv}{dt}\\
            &= m\sqrt{\frac{\gamma}{m}}\\
            \Rightarrow F &= \sqrt{m\gamma}
            \intertext{Therefore, the force applied on the particle is constant. Option (a) is correct.}
            \intertext{The speed of the particle is proportional to time. Option (b) is correct.}
            \intertext{For distance, we can integrate speed with respect to time:}
            x &= \int_0^t v dt\\
            &= \int_0^t \sqrt{\frac{\gamma}{m}} t dt\\
            &= \sqrt{\frac{\gamma}{m}} \int_0^t t dt\\
            &= \sqrt{\frac{\gamma}{m}} \left[\frac{t^2}{2}\right]_0^t\\
            \Rightarrow x &= \sqrt{\frac{\gamma}{m}} \frac{t^2}{2}
            \intertext{Therefore, the distance of the particle from the origin increases quadratically with time. Option (c) is incorrect.}
            \intertext{To check if the force is conservative, we can check if the work done by the force is path independent.}
            \intertext{The work done by the force is given by:}
            W &= \int \vec{F} \cdot d\vec{r}\\
            &= \int_{x_i}^{x_f} F dx\\
            &= \int_{x_i}^{x_f} \sqrt{m\gamma} dx\\
            &= \sqrt{m\gamma} \int_{x_i}^{x_f} dx\\
            &= \sqrt{m\gamma} \Delta x
            \intertext{Therefore, the work done by the force is path independent. Option (d) is correct.}
            \intertext{Therefore, the correct options are (a), (b) and (d).}
        \end{align*}
    \end{solution}
