
    \item A steel wire of diameter \(0.5 \, \text{mm}\) and Young's modulus \(2 \times 10^{11} \, \text{N/m}^2\) carries a load of mass \(M\). The length of the wire with the load is \(1.0 \, \text{m}\). A vernier scale with \(10\) divisions is attached to the end of this wire. Next to the steel wire is a reference wire to which a main scale, of least count \(1.0 \, \text{mm}\), is attached. The \(10\) divisions of the vernier scale correspond to \(9\) divisions of the main scale. Initially, the zero of vernier scale coincides with the zero of main scale. If the load on the steel wire is increased by \(1.2 \, \text{kg}\), the vernier scale division which coincides with a main scale division is \_\_\_\_\_\_. Take \(g = 10 \, \text{m/s}^2\) and \(\pi = 3.2\).
