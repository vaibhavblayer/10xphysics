\item A planet of mass \( M \), has two natural satellites with masses \( m_1 \) and \( m_2 \). The radii of their circular orbits are \( R_1 \) and \( R_2 \) respectively. Ignore the gravitational force between the satellites. Define \( v_1 \), \( L_1 \), \( K_1 \) and \( T_1 \) to be, respectively, the orbital speed, angular momentum, kinetic energy and time period of revolution of satellite 1; and \( v_2 \), \( L_2 \), \( K_2 \) and \( T_2 \) to be the corresponding quantities of satellite 2. Given \( m_1/m_2 = 2 \) and \( R_1/R_2 = 1/4 \), match the ratios in List-I to the numbers in List-II.

\begin{center}
    \renewcommand{\arraystretch}{2.5}
    \begin{table}
        \centering
        \begin{tabular}{p{0.25cm}p{6cm}|p{0.25cm}p{6cm}}
        \hline
        & List-I & &List-II \\
        \hline
        P. & \( \frac{v_1}{v_2} \) & 1. & \( \frac{1}{8} \) \\
        Q. & \( \frac{L_1}{L_2} \) & 2. & \( 1 \) \\
        R. & \( \frac{K_1}{K_2} \) & 3. & \( 2 \) \\
        S. & \( \frac{T_1}{T_2} \) & 4. & \( 8 \) \\
        \hline
        \end{tabular}
    \end{table}
\end{center}

\begin{tasks}(2)
    \task \( P \rightarrow 4; \) \( Q \rightarrow 2; \) \( R \rightarrow 1; \) \( S \rightarrow 3 \)
    \task \( P \rightarrow 3; \) \( Q \rightarrow 2; \) \( R \rightarrow 4; \) \( S \rightarrow 1 \)
    \task \( P \rightarrow 2; \) \( Q \rightarrow 3; \) \( R \rightarrow 1; \) \( S \rightarrow 4 \)
    \task \( P \rightarrow 2; \) \( Q \rightarrow 3; \) \( R \rightarrow 4; \) \( S \rightarrow 1 \)
\end{tasks}

\begin{solution}
    Since the satellites are in circular orbits, we can use the following equations governing their motion:
    \begin{align*}
        v &= \sqrt{\frac{GM}{R}} \quad \text{(Orbital speed)} \\
        L &= mRv \quad \text{(Angular momentum)} \\
        K &= \frac{1}{2}mv^2 \quad \text{(Kinetic energy)} \\
        T &= \frac{2\pi R}{v} \quad \text{(Time period)}
    \end{align*}
    where \( G \) is the gravitational constant.
    \begin{align*}
        \intertext{For the ratio \( P = \frac{v_1}{v_2} \), we use the orbital speed equation:}
        v_1 &= \sqrt{\frac{GM}{R_1}} \\
        v_2 &= \sqrt{\frac{GM}{R_2}} \\
        \frac{v_1}{v_2} &= \sqrt{\frac{R_2}{R_1}} = \sqrt{\frac{4}{1}} = 2 \quad \Rightarrow P \rightarrow 3.
        \intertext{For the ratio \( Q = \frac{L_1}{L_2} \), we use the angular momentum equation:}
        L_1 &= m_1R_1v_1 \\
        L_2 &= m_2R_2v_2 \\
        \frac{L_1}{L_2} &= \frac{m_1R_1}{m_2R_2}\frac{v_1}{v_2} = \frac{2}{1}\frac{1}{4}\cdot2 = \frac{1}{1} = 1 \quad \Rightarrow Q \rightarrow 2.
        \intertext{For the ratio \( R = \frac{K_1}{K_2} \), we use the kinetic energy equation:}
        K_1 &= \frac{1}{2}m_1v_1^2 \\
        K_2 &= \frac{1}{2}m_2v_2^2 \\
        \frac{K_1}{K_2} &= \frac{m_1}{m_2}\left(\frac{v_1}{v_2}\right)^2 = \frac{2}{1}\cdot2^2 = 8 \quad \Rightarrow R \rightarrow 4.
        \intertext{Finally, for the ratio \( S = \frac{T_1}{T_2} \), we use the time period equation:}
        T_1 &= \frac{2\pi R_1}{v_1} \\
        T_2 &= \frac{2\pi R_2}{v_2} \\
        \frac{T_1}{T_2} &= \frac{R_1}{R_2}\frac{v_2}{v_1} = \frac{1}{4}\cdot\frac{1}{2} = \frac{1}{8} \quad \Rightarrow S \rightarrow 1.
    \end{align*}
    Thus, we have P matching with 3, Q with 2, R with 4, and S with 1. The final answer is
    \begin{align*}
        P \rightarrow 3; \quad Q \rightarrow 2; \quad R \rightarrow 4; \quad S \rightarrow 1.\\
        \intertext{Therefore, option (b) is correct.}
    \end{align*}
\end{solution}