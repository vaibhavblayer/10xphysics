
    \item In a photoelectric experiment a parallel beam of monochromatic light with power of \(200 \, \text{W}\) is incident on a perfectly absorbing cathode of work function \(6.25 \, \text{eV}\). The frequency of light is just above the threshold frequency so that the photoelectrons are emitted with negligible kinetic energy. Assume that the photoelectron emission efficiency is \(100\%\). A potential difference of \(500 \, \text{V}\) is applied between the cathode and the anode. All the emitted electrons are incident normally on the anode and are absorbed. The anode experiences a force \(F = n \times 10^{-4} \, \text{N}\) due to the impact of the electrons. The value of \(n\) is \_\_\_\_\_\_\_\_. Mass of the electron \(m_e = 9 \times 10^{-31} \, \text{kg}\) and \(1.0 \, \text{eV} = 1.6 \times 10^{-19} \, \text{J}\).
