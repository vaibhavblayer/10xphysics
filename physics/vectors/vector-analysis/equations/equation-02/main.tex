\documentclass{article}
\usepackage{v-equation}
\geometry{
paperwidth=5in, 
paperheight=5in, 
top=15mm, 
bottom=15mm, 
left=10mm, 
right=10mm
}

\begin{document}
\ttfamily
\begin{center}
Geometrical meaning of Curl
\end{center}
\vspace*{\fill}
\begin{center}
\begin{tikzpicture}
\fill (0, 0) circle(2pt);
\foreach \a in {10, 40, ..., 340}{
\draw[->] (0.5*cos{\a}, 0.5*sin{\a})--(cos{\a}, sin{\a});
\draw[->] (1.25*cos{\a}, 1.25*sin{\a})--(2*cos{\a}, 2*sin{\a});
}
\begin{scope}[xshift=3.5cm, yshift=-1cm]
\foreach \x in {0, 0.4, ..., 4}{
\draw[->] (\x, 0)--(\x, 0.35);
\draw[->] (\x, 0.5)--(\x, 1);
\draw[->] (\x, 1.15)--(\x, 2);
}
\end{scope}
\end{tikzpicture}
\end{center}
\begin{align*}
\nabla \times \text{vector} = 0
\end{align*}

Curl is a measure of how much the vector swirls around the point.
In the above figure curl is zero.

\vspace*{\fill}
\end{document}
