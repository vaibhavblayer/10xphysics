\documentclass{article}
\usepackage{v-problem}
\vgeometry

\begin{document}
\vtitle[ROTATION]

\def\pn{01}
\def\book{A.K.}
\def\page{39}
\def\gdrive{https://drive.google.com/drive/folders/1au0dT1CE55FT5JG2AbQQlbXIadA604rO?usp=share_link}

\def\question{
A bobbin with thread wound around it lies on a horizontal floor and can roll along it without sliding. With what velocity will the center of the bobbin move if the block starts to descend with constant velocity $v$ as shown in the figure ? The inner radius of the bobbin is $r$ and external one $R$.
}

\vspace*{\fill}
\begin{tikzpicture}
	\node[qnumber] (n) at (0, 0)[scale=2] {$\pn.$};
	\node[question] (q) [right=2mm of n.east] {\question};
	\tzline[divider]<-0.125, 0> (q.north west)(q.south west);
	\node[format] (f) at  (q.south east){[\book \quad \page]};
\end{tikzpicture}	
\vspace*{\fill}

\begin{center}
\begin{tikzpicture}
\def\r{0.65}
\def\R{1.2}
\pic at (0, 0) {frame=5cm};
\pic[rotate=-90] at (2.5, -1.25) {frame=2.5cm};
\tzring*[pattern=north east lines](-1,\R)(\r)(-1,\R)(\R)
\fill (-1, \R) circle(2pt);
\tzline+[->](-1, \R)(-20:\r){$r$}[mb]
\tzline+[->](-1, \R)(40:\R){$R$}[mb]
\tzcircle[fill=black!55](2.5, 0)(\R-\r)
\fill (2.5, 0) circle(2pt);
\tzline(-1, \R-\r)(2.5, \R-\r)
\tzline+(2.5+\R-\r, 0)(0, -1)
\node[block, xshift=\R cm-\r cm, yshift=-1.4cm]  at (2.5, 0) {$m$};
\tzline+[->](4, -1)(0, -1) {$v$}[mr]
\end{tikzpicture}
\end{center}
\pagebreak

\vtitle[\texttt{Solution}]

\begin{center}
\begin{tikzpicture}
\def\r{0.65}
\def\R{1.2}
\begin{scope}
	\tzring*[pattern=north east lines](0,0)(\r)(0, 0)(\R)
	\tzline+[->](0, -\r)(1.5, 0){$v$}[r]
	\tzline+[->](0, -\R)(1.5, 0){$v_p$}[r]
	\tzdot*(0, -\r){$I$}[l]
	\tzdot*(0, -\R){$P$}[b]
	\tzdot*(0, 0){$O$}[b]
	\tzarc'[->](0, 0)(-150:30:0.25){$\omega$}[ma]
\end{scope}
\begin{scope}[xshift=4cm]
	\node[block, ] (mass) at (0, 0) {$m$};
	\tzline+[->](mass.south)(0, -1){$v$}[b]
\end{scope}
\end{tikzpicture}
\end{center}

\begin{align}
\intertext{As the bobbin is in pure rolling motion it's motion will be pure rotational aboout the instantaneous axis of rotation which passes through point of contact $P$.}
v &= (R-r)\omega \\
\intertext{Velocity of point $O$ w.r.t. point of contact $P$.}
v_O &= R\omega
\intertext{From equation (1) and (2) we have}
\Aboxed{v_O &= \dfrac{R}{(R-r)} v} \ans
\end{align}

\pagebreak

\vspace*{\fill}
\begin{center}
	\fbox{\qrcode[height=2cm]{\gdrive}}
\end{center}
\vspace*{\fill}

\end{document}
