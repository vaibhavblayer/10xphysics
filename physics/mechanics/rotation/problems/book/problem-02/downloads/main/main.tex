\documentclass{article}
\usepackage{v-test-paper}

\renewcommand{\frac}[2]{\dfrac{#1}{#2}}


\begin{document}



\begin{enumerate}
    \item Prove the relation, \(s_t = u + at - \frac{1}{2} a\).
    \item Equation \(s_t = u + at - \frac{1}{2} a\) does not seem dimensionally correct, why?
    \item A particle is projected vertically upwards. What is the value of acceleration
    \begin{enumerate}
        \item during upward journey,
        \item during downward journey and
        \item at highest point?
    \end{enumerate}
    \item A ball is thrown vertically upwards. Which quantity remains constant among, speed, kinetic energy, velocity and acceleration?
    \item A particle is projected vertically upwards with an initial velocity of 40 m/s. Find the displacement and distance covered by the particle in 6 s. Take \(g = 10 m/s^2\).
    \item A particle moves rectilinearly with initial velocity \(u\) and constant acceleration \(a\). Find the average velocity of the particle in a time interval from \(t = 0\) to \(t = t\) second of its motion.
    \item A particle moves in a straight line with uniform acceleration. Its velocity at time \(t = 0\) is \(v_1\) and at time \(t = t\) is \(v_2\). The average velocity of the particle in this time interval is \(\frac{v_1 + v_2}{2}\).
    
    Is this statement true or false?
    \item Find the average velocity of a particle released from rest from a height of 125 m over a time interval till it strikes the ground. Take \(g = 10 m/s^2\).
    \item A particle starts with an initial velocity 2.5 m/s along the positive x-direction and it accelerates uniformly at the rate 0.50 m/s\(^2\).
    \begin{enumerate}
        \item Find the distance travelled by it in the first two seconds
        \item How much time does it take to reach the velocity 7.5 m/s?
        \item How much distance will it cover in reaching the velocity 7.5 m/s?
    \end{enumerate}
    \item A ball is projected vertically upward with a speed of 50 m/s. Find
    \begin{enumerate}
        \item the maximum height,
        \item the time to reach the maximum height,
        \item the speed at half the maximum height. Take \(g = 10 m/s^2\).
    \end{enumerate}
\end{enumerate}


\begin{enumerate}
    \item Velocity (in \(m/s\)) of a particle moving along x-axis varies with time as, \(v = (10 + 5t - t^2)\) At time \(t = 0, x = 0\). Find
    \begin{enumerate}
        \item acceleration of particle at \(t = 2 s\) and
        \item x-coordinate of particle at \(t = 3 s\)
    \end{enumerate}
    \item A particle is moving with a velocity of \(v = (3 + 6t + 9t^2) cm/s\). Find out
    \begin{enumerate}
        \item the acceleration of the particle at \(t = 3 s\).
        \item the displacement of the particle in the interval \(t = 5 s\) to \(t = 8 s\).
    \end{enumerate}
    \item The motion of a particle along a straight line is described by the function \(x = (2t - 3t^2)\), where \(x\) is in metres and \(t\) is in seconds. Find
    \begin{enumerate}
        \item the position, velocity and acceleration at \(t = 2 s\).
        \item the velocity of the particle at origin.
    \end{enumerate}
    \item x-coordinate of a particle moving along this axis is \(x = (2 + t^2 + 2t^3)\). Here, \(x\) is in metres and \(t\) in seconds. Find
    \begin{enumerate}
        \item position of particle from where it started its journey,
        \item initial velocity of particle and
        \item acceleration of particle at \(t = 2 s\).
    \end{enumerate}
    \item The velocity of a particle moving in a straight line is directly proportional to \(3/4\)th power of time elapsed. How does its displacement and acceleration depend on time?
\end{enumerate}

\pagebreak

\begin{enumerate}
    \setcounter{enumi}{31}
      \item Consider 2.00 mole of an ideal diatomic gas. Find the total heat capacity at constant volume and at constant pressure if (a) the molecules rotate but do not vibrate, and (b) if the molecules both rotate and vibrate.\\[1mm]
      [Answer: (a) 29.5 cal/K, 13.9 cal/K, (b) 13.9 cal/K, 17.9 cal/K]
      \item In a crude model (Fig. 1.c.33) of a rotating diatomic molecule of chlorine (\( \text{Cl}_2 \)), the two Cl atoms are 2.00 x \( 10^{-10} \) m apart and rotate about their centre of mass with angular speed \( \omega = 2.00 \times 10^{12} \) rad/s. What is the rotational kinetic energy of one molecule of \( \text{Cl}_2 \), which has a molar mass of 70.0 g/mol?
    \begin{tikzpicture}
    % Diagram for 33 can be drawn here with TikZ
    \end{tikzpicture}
      \item The compressibility \( \kappa \) of a substance is defined as the fractional change in volume of that substance for a given change in pressure:
      \[
      \kappa = -\frac{1}{V} \left( \frac{\partial V}{\partial P} \right)
      \]
      \begin{enumerate}
        \item Explain why the negative sign in this expression ensures that \( \kappa \) is always positive.
        \item Show that if an ideal gas is is always compressed isothermally, its compressibility is given by \( \kappa_1 = 1/P \).
        \item Show that if an ideal gas is compressed adiabatically, its compressibility is given by \( \kappa_2 = 1/\gamma P \).
        \item Determine values for \( \kappa_1 \) and \( \kappa_2 \) for a monatomic ideal gas at a pressure of 2.00 atm.
      \end{enumerate}
      [Answer: (d) 0.5 atm\(^{-1}\), 0.3 atm\(^{-1}\)]
      \item Show that the speed of sound in an ideal gas is
      \[
      v = \sqrt{\frac{\gamma RT}{M}}
      \]
      \item Sixteen identical molecules have various speeds: one has a speed of 2.00 m/s; two have a speed of 3.00 m/s; three have a speed of 5.00 m/s; four have a speed of 7.00 m/s; three have a speed of 9.00 m/s and two have a speed of 12.0 m/s. Find (a) the average speed, (b) the rms speed, and (c) the most probable speed of these particles.
      [Answer: (a) 6.80 m/s, (b) 7.41 m/s, (c) 7 m/s]
      \item A gas is at \( 0^\circ C \). If we wish to double the rms speed of the gas's molecules, by how much must we raise its temperature?
      [Answer: 819\(^{\circ}\)C]
      \item A container has a mixture of two gases: \( n_1 \) moles of gas 1, which has a molar specific heat \( C_1 \), and \( n_2 \) moles of gas 2, which has a molar specific heat \( C_2 \). What is the molar specific heat of the mixture. [Answer: \( C_{\text{mix}} = \frac{n_1 C_1 + n_2 C_2}{n_1 + n_2} \)]
      \item Air consists mainly of \( N_2 \) and \( O_2 \), mixed in the ratio 4: 1. The atomic mass of N is 14 u, and that of O is 16 u. Find the mean molecular mass of air.
      \item At 1 atm pressure and 100\(^{\circ}\)C, a given mass of steam occupies 1672 times the volume occupied by an equal mass of water. What is the approximate ratio of the mean distance between the water molecules in steam to the diameter of the water molecules?
      \item Suppose that you make a plot on log-log paper of \( v_{\text{rms}} \) versus \( T \) for an ideal gas. (Alternatively, you could plot \( \ln v_{\text{rms}} \) versus \( \ln T \) on ordinary graph paper.) Show that the plot is a straight line having slope 1/2.
      \item Estimate the molar heat capacity for a gas whose molecules have (a) seven degrees of freedom, (b) twelve degrees of freedom.
      \item Oxygen gas is contained in a cylinder whose volume is 0.0235 \( m^3 \). The total translational kinetic energy of the molecules in the gas is 5.03 x \( 10^3 \) J, and the rms speed of the molecules is 1.97 x \( 10^3 \) m/s. (a) Find the mass of the oxygen. (b) What is the pressure inside the cylinder? (c) What is the total rotational kinetic energy of the molecules of the gas?
      \item A space vehicle returning from the Moon enters the atmosphere at a speed of about 40,000 km/h. Molecules (assume nitrogen) striking the nose of the vehicle with this speed correspond to what temperature? (Because of this high temperature, the nose of a space vehicle must be made of special materials; indeed, part of it does vaporize, and this is seen as a bright blaze upon re-entry.)
      \item In an ultrahigh vacuum system, the pressure is measured to be \( 1.0 \times 10^{-10} \) torr (where 1 torr = 133 Pa). Assume that the gas molecules have a molecular diameter of 3.00 x \( 10^{-10} \) m and that the temperature is 300K. Find (a) the number of molecules in a volume of 1.00 \( m^3 \), (b) the mean free path of the molecules, and (c) the collision frequency, assuming an average speed of 500 \( m/s \).
      \item A liquid is enclosed in a metal cylinder that is provided with a piston of the same metal. The system is originally at a pressure of 100 atm (1.013 x \( 10^5 \) Pa) and at a temperature of 30.0\(^{\circ}\)C. The piston is forced down until the pressure on the liquid is increased by \ldots
    \end{enumerate}

\end{document}