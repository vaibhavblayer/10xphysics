\documentclass{article}
\usepackage{v-equation}

\begin{document}
\Large
\def\gdrive{https://drive.google.com/drive/folders/1bRnGnog8GUPpYx0sVQJV12olKyDwaWqj?usp=share_link}

\vtitle[radius of curvature]
\begin{center}
\begin{tikzpicture}
[thick]
\def\tat{3.65}%tangent at {x}
\def\dx{0.25}%tangent at {x}
	\tzaxes(-0.5, -0.5)(7,4){$x$}{$f(x)$}
	\tztos"curve"
		(0,0)[out=0, in=-180]
		(3,1.5)[out=0, in=-180]
		(6, 3.5);
		
	\tzvXpointat{curve}{\tat-\dx}(A)
	\tzvXpointat{curve}{\tat+\dx}(B)
	\tztangentat"TL"{curve}{\tat}[\tat-1.5:5]
	\tzcoor*($($(A)!0.5!(B)$)!1cm!90:(B)$)(O){$O$}[ar]
	%\tzXpoint{normal}{normall}(X)
	\tzcircle[dashed](O)(1)
	\tzline[dashed, red](O)(A)
	\tzline[dashed, red](O)(B)
	\tzcoor*($(A)+(2*\dx, 0)$)(C)
	\tzline[red](A)(C)
	\tzline[red](B)(C)
\end{tikzpicture}
\end{center}
\vspace*{\fill}
\begin{align*}
r &= \dfrac{\left[1 + \left(\dfrac{\d{y}}{\d{x}}\right)^2 \right]^{3/2}}{\left| \dfrac{\d{^2y}}{\d{x^2}} \right|}
\end{align*}
\vspace*{\fill}

\pagebreak

\begin{center}
\begin{tikzpicture}
[thick, scale=3.5]
\clip (2, 1) rectangle (4.25, 4);
\def\tat{3.65}%tangent at {x}
\def\dx{0.2}%tangent at {x}
	\tzaxes(-0.5, -0.5)(7,4){$x$}{$f(x)$}
	\tztos"curve"
		(0,0)[out=0, in=-180]
		(3,1.5)[out=0, in=-180]
		(6, 3.5);
		
	\tzvXpointat*{curve}{\tat-\dx}(A)
	\tzvXpointat*{curve}{\tat+\dx}(B)
	\tztangentat"TL"{curve}{\tat}[\tat-1.5:5]
	\tzcoor*($($(A)!0.5!(B)$)!1cm!90:(B)$)(O){$O$}[ar]
	%\tzXpoint{normal}{normall}(X)
	\tzcircle[dashed](O)(1)
	\tzline[dashed, red](O)(A){$r$}[ml]
	\tzline[dashed, red](O)(B)
	\tzcoor*($(A)+(2*\dx, 0)$)(C)
	\tzline[red](A)(C){$\d{x}$}[mb]
	\tzline[red](B)(C){$\d{y}$}[mr]
	\tzline[red](A)(B){$\d{l}$}[ma]
	\tzanglemark(C)(A)(B){$\theta$}(5pt)
	\tzanglemark(B)(O)(A){$\d{\theta}$}(5pt)
\end{tikzpicture}
\end{center}

\addtolength{\jot}{2ex}
\begin{align}
r \d{\theta} &= \d{l} \\
\tan\theta &= \dfrac{\d{y}}{\d{x}}\\
\d{l}^2 &= \d{x}^2 + \d{y}^2
\end{align}
From equation (2)\_
\begin{align}
\sec^2\theta \dfrac{\d{\theta}}{\d{x}} &= \dfrac{\d{^2y}}{\d{x^2}}
\end{align}
From equation (3)\_
\begin{align}
\dfrac{\d{l}}{\d{x}} &= \sqrt{1+ \left( \dfrac{\d{y}}{\d{x}} \right)^2}   
\end{align}

From equation (4) and (5)\_
\begin{align}
\dfrac{\d{l}}{\d{\theta}} &= \dfrac{\sqrt{1+ \left( \dfrac{\d{y}}{\d{x}} \right)^2}}{\dfrac{\d{^2y}}{\d{x^2}}} \sec^2\theta
\end{align}


Now, from equation (1) and (6)\_
\begin{align*}
r &= \dfrac{\d{l}}{\d{\theta}}\\
	&= \dfrac{\sqrt{1+ \left( \dfrac{\d{y}}{\d{x}} \right)^2}}{\dfrac{\d{^2y}}{\d{x^2}}} \sec^2\theta \\
	&= \dfrac{\sqrt{1+ \left( \dfrac{\d{y}}{\d{x}} \right)^2}}{\dfrac{\d{^2y}}{\d{x^2}}} \left( 1 + \tan^2\theta\right) \\
	&=\dfrac{\left[1 + \left(\dfrac{\d{y}}{\d{x}}\right)^2 \right]^{3/2}}{\left| \dfrac{\d{^2y}}{\d{x^2}} \right|}
\end{align*}

\end{document}
