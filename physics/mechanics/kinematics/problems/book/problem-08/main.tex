\documentclass{article}
\usepackage{v-problem}
\vgeometry

\begin{document}
\vtitle[KINEMATICS]

\def\pn{08}
\def\book{A.K.}
\def\page{35}
\def\gdrive{https://drive.google.com/drive/folders/1H_uWnU3LXCRthLukKkt-fnVWMcX-T2oy?usp=share_link}

\def\question{
A hinged construction consists of three rhombus with the ratio of sides $3:2:1$ (Fig.). Vertex $A_3$ moves in the horizontal direction at a velocity $v$. Determine the velocities of vertices $A_1$ ,$A_2$ and $B_2$ at the instant when the angles of the construction are $90^\circ$.
}

\vspace*{\fill}
\begin{tikzpicture}
	\node[qnumber] (n) at (0, 0)[scale=2] {$\pn.$};
	\node[question] (q) [right=2mm of n.east] {\question};
	\tzline[divider]<-0.125, 0> (q.north west)(q.south west);
	\node[format] (f) at  (q.south east){[\book \quad \page]};
\end{tikzpicture}	
\vspace*{\fill}

\begin{center}
\begin{tikzpicture}
[thick]
\pic[rotate=-90] at (0, 0) {frame=6cm};
\foreach \s/\x/\n in {3/0/1, 2/3/2, 1/5/3}{
	\draw[xshift=\x cm] (0, 0)
		--++(-60:\s)node[below]{$C_\n$}
		--++(60:\s) node[right]{$A_\n$}
		--++(120:\s)node[above]{$B_\n$}-- cycle;
}
\tznode(0, 0){$A_0$}[r]
\tzline+[->](6, 0)(1.5, 0){$v$}[r]
\end{tikzpicture}
\end{center}
\pagebreak

\vtitle[Solution]
Let us take point $A_0$ as origin and indicate x-axis as shown in the Fig. below. If the ratio constant of the sides is denoted by $k$, then obviously at an arbitrary moment when side $A_0B_1$, $A_1B_2$ or $A_2B_3$ makes an angle $\theta$ with the horizontal we have :

\begin{center}
\begin{tikzpicture}
[thick]
\pic[rotate=-90] at (0, 0) {frame=6cm};
\foreach \s/\x/\n in {3/0/1, 2/3/2, 1/5/3}{
	\draw[xshift=\x cm] (0, 0)
		--++(-60:\s)node[below]{$C_\n$} 
		--++(60:\s) node[right]{$A_\n$} coordinate (A\n)
		--++(120:\s)node[above]{$B_\n$} coordinate (B\n)-- cycle;
}
\tzcoor*(0,  0)(A0){$A_0$}[r]
\tzline+[->](A3)(1.5, 0){$v$}[r]
\tzline[dashed](A0)(A3)
\tzanglemark(A1)(A0)(B1){$\theta$}(15pt)
\tzdot*(A1)
\tzline+[->](A1)(1, 0){$\vec{v}_{A_1}$}[a]
\tzdot*(A2)
\tzline+[->](A2)(1, 0){$\vec{v}_{A_2}$}[a]
\tzdot*(B2)
\tzline+[->](B2)(1, 0){$\vec{v}_{B_2}$}[ar]
\end{tikzpicture}
\end{center}
\addtolength{\jot}{2ex}
\begin{align*}
A_0B_1 = 3k \qquad A_1B_2 = 2k  \qquad A_2B_3=k
\end{align*}

\pagebreak
\begin{align*}
x_{A_1} = 6k \cos\theta \qquad  x_{A_2} = 10k \cos\theta \qquad x_{A_3} = 12k  \cos\theta \\ x_{B_2} = 8k \cos\theta \qquad y_{B_2} = 2k \sin\theta
\end{align*}

\begin{align*}
\dfrac{\d{x_{A_3}}}{\d{t}} &= v\\
-12k\sin\theta\dfrac{\d{\theta}}{\d{t}} &= v\\
\Aboxed{-k\sin\theta\dfrac{\d{\theta}}{\d{t}} &= \dfrac{v}{12}}\\[5mm]
\end{align*}
	
Now we can calculate $v_{A_1}$, $v_{A_2}$ and $v_{B_2}$ using above equation\_
\begin{itemize}
\item
\begin{align*}
v_{A_1} &= \dfrac{\d{x_{A_1}}}{\d{t}} \\
		&=-6k\sin\theta\dfrac{\d{\theta}}{\d{t}}\\
		&= 6 \cdot \dfrac{v}{12}\\
\Aboxed{v_{A_1} &= \dfrac{v}{2}}\ans
\end{align*}
\item
\begin{align*}
v_{A_2} &= \dfrac{\d{x_{A_2}}}{\d{t}} \\
		&=-10k\sin\theta\dfrac{\d{\theta}}{\d{t}}\\
		&= 10 \cdot \dfrac{v}{12}\\
\Aboxed{v_{A_2} &= \dfrac{5v}{6}}\ans 
\end{align*}
\item $B_2$ is moving forward and upward both\_
\begin{align*}
v_{x_{B_2}} &= \dfrac{\d{x_{B_2}}}{\d{t}} \\
		&=-8k\sin\theta\dfrac{\d{\theta}}{\d{t}}\\
		&= 8 \cdot \dfrac{v}{12}\\
\Aboxed{v_{x_{B_2}} &= \dfrac{2v}{3}}
\end{align*}
\begin{align*}
v_{y_{B_2}} &= \dfrac{\d{y_{B_2}}}{\d{t}} \\
		&=-2k\sin\theta\dfrac{\d{\theta}}{\d{t}}\\
		&= 2 \cdot \dfrac{v}{12}\\
\Aboxed{v_{y_{B_2}} &= \dfrac{v}{6}}
\end{align*}

\begin{align*}
v_{B_2} &= \sqrt{v_{x_{B_2}}^2 + v_{y_{B_2}}^2} \\
		&=\sqrt{\left(\dfrac{2v}{3}\right)^2 + \left(\dfrac{v}{6}\right)^2} \\
		&= \sqrt{\dfrac{4v^2}{9} + \dfrac{v^2}{36}}\\
\Aboxed{v_{B_2} &= \dfrac{\sqrt{17}}{6}v}\ans
\end{align*}

\end{itemize}

\pagebreak
\vspace*{\fill}
\begin{center}
	\fbox{\qrcode[height=2cm]{\gdrive}}
\end{center}
\vspace*{\fill}

\pagebreak
\vspace*{\fill}

\begin{center}
\begin{tikzpicture}
[thick]
	\tzwedge(0, 0)(-60:240:2){\texttt{how capable you really are}}[ma]
	\tzwedge<0, -0.2>(0, 0)(240:300:2){\texttt{how capable you think you are}}[mb]
\end{tikzpicture}
\end{center}
\vspace*{\fill}

\end{document}
