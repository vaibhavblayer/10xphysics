\documentclass{article}
\usepackage{v-problem}
\usepackage{v-temp-macro}
\vgeometry

\begin{document}
\vtitle[KINEMATICS]

\def\pn{01}
\def\exam{IIT-JEE}
\def\year{2016}
\def\gdrive{https://drive.google.com/drive/folders/1MyNIyEawU1Ote8-yj1a2Tujg45iw-hrE?usp=share_link}

\def\question{
The position vector $\vec{r}$ of particle of mass $m$ is given by the
following equation \[ \vec{r}(t)=\alpha t^3\hat{i} + \beta t^2\hat{j} \]
where, $\alpha=\dfrac{10}{3}ms^{-3}, \beta=5ms^{-2} \text{ and } m=0.1\kg $. At $t=1\s$, which of the following statement(s) is(are) true about the particle? 
}

\def\option{
\begin{tasks}(1)
\task The velocity $\vec{v}$ is given by $\vec{v}=\left(10\hat{i}+10\hat{j}\right)\mps$ \ans
\task The angular momentum $\vec{L}$ with respect to the origin is given by $\vec{L}=\dfrac{5}{3}\hat{k}\N\m\s$
\task The force $\vec{F}$ is given by $\vec{F}=\left(\hat{i}+2\hat{j}\right)\N$
\task The torque $\tau$ with respect to the origin is given by $\tau=-\dfrac{20}{3}\hat{k}\N\m$ \ans
\end{tasks}
}

\vspace*{\fill}
\begin{assemble}[S]
	\node[qnumber, \C] (n) at (0, 0)[scale=2] {$\pn.$};
	\node[question] (q) [right=2mm of n.east] {\question};
	\tzline[divider, \C]<-0.125, 0> (q.north west)(q.south west);
	\node[format] (f) at  (q.south east){[\exam \quad \year]};
\end{assemble}	
\vspace*{\fill}

\begin{tikzpicture}
\node[minimum width=1cm](n) at (0, 0){}; 
\node[option, anchor=west] at (n.east){\option};
\end{tikzpicture}
\vspace*{\fill}

\pagebreak

\begin{align}
\vec{r}(t) &=\alpha t^3\hat{i} + \beta t^2\hat{j}\\
\vec{v}(t) &= 3\alpha t^2\hat{i} + 2\beta t\hat{j}\\
\vec{a}(t) &= 6\alpha t\hat{i} + 2\beta \hat{j}
\end{align}

\begin{enumerate}[label=(\alph*)]
\item From equation (2)
\begin{align*}
\vec{v}(1) &= 3\cdot\dfrac{10}{3}\cdot 1^2\hat{i} + 2\cdot 5 \cdot 1\hat{j}\\
&= 10\hat{i} + 10\hat{j} \ans
\end{align*}

\item Using equation (1) and (2)

\begin{align*}
\vec{L} &= \vec{r} \times \vec{p}\\
		&= \left(\alpha t^3 \hat{i} + \beta t^2\hat{j} \right) \times \left(0.1\right)\left(3\alpha t^2\hat{i} + 2\beta t\hat{j}\right)\\
		&= -0.1\alpha\beta t^4\hat{k}\\
		&= -\dfrac{5}{3}\hat{k}
\end{align*}

\item Using equation (3)
\begin{align*}
\vec{F} &= m\vec{a}\\
		&= 0.1 \left( 6\alpha t\hat{i} + 2\beta\hat{j} \right)\\
		&= 2\hat{i} + \hat{j}
\end{align*}

\item From (b) we can use $\vec{L}$
\begin{align*}
\vec{\tau} &= \dfrac{\d{\vec{L}}}{\d{t}} \\
		&= \dfrac{\d{}}{\d{t}}\left(-0.1\alpha\beta t^4 \hat{k} \right)\\
		&= -0.1\alpha\beta \cdot 4t^3\hat{k}\\
		&= -\dfrac{20}{3}\hat{k} \ans
\end{align*}
\end{enumerate}



\vspace*{\fill}
\begin{center}
	\fbox{\qrcode[height=2cm]{\gdrive}}
\end{center}
\vspace*{\fill}

\end{document}
