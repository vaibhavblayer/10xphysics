\item An aerostat of mass \( m \) starts coming down with a constant acceleration \( w \). Determine the ballast mass to be dumped for the aerostat to reach the upward acceleration of the same magnitude. The air drag is to be neglected.


\begin{solution}
    \begin{align*}
        \intertext{Let R be the constant upward thrust on the aerostat of mass  m, coming down with a constant acceleration w. Applying Newton’s second law of motion for the aerostat in projection form} 
        F_y &= mw_y \\
        mg - R &= mw \tag{1} \\
        \\
        \intertext{Now, if $\Delta m$ be the mass to be dumped, then using the equation $F_y = mw_y$} 
        R - (m - \Delta m) g &= (m - \Delta m) w \tag{2} \\
        \\
        &\text{From Eqs. (1) and (2), we get} \\
        \Delta m &= \dfrac{2mw}{g + w}
    \end{align*}
\end{solution}
