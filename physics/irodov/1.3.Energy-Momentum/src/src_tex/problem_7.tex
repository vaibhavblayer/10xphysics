\item A chain of mass \( m = 0.80 \, \text{kg} \) and length \( l = 1.5 \, \text{m} \) rests on a rough-surfaced table so that one of its ends hangs over the edge. The chain starts sliding off the table all by itself provided the overhanging part equals \( \eta = \frac{1}{3} \) of the chain length. What will be the total work performed by the friction forces acting on the chain by the moment it slides completely off the table?\begin{solution}
    \begin{center}
        \begin{tikzpicture}
            \pic at (0, 0) {frame=3cm};
        \end{tikzpicture}
    \end{center}
    
    \begin{align*}
        \intertext{From the initial condition of the problem, the limiting friction between the chain lying on the horizontal table equals the weight of the over hanging part of the chain, i.e.,}
        \lambda \eta lg &= k \lambda (1 - \eta) lg \quad (\text{where } \lambda \text{ is the linear mass density of the chain}).\\
        \intertext{So,}
        k &= \dfrac{\eta}{1 - \eta} \tag{1}\\
        \intertext{Let (at an arbitrary moment of time) the length of the chain on the table be $x$. So the net friction force between the chain and the table, at this moment}
        fr &= k \lambda x g = \lambda x g\\
        \intertext{The differential work done by the friction forces}
        dA &= fr \cdot d\vec{r} = - fr \, ds\\
        &= -k \lambda x g \, (-dx) = \lambda g \left( \dfrac{\eta}{1 - \eta} \right) x \, dx \tag{3}\\
        \intertext{(Note that here we have written $ds = -dx$, because $ds$ is essentially a positive term and as the length of the chain decreases with time, $dx$ is negative.)}
        \intertext{Hence, the sought work done}
        A &= \int_{(1 - \eta) l}^{0} \lambda g \dfrac{\eta}{1 - \eta} x \, dx = -(1 - \eta)\eta \dfrac{mgl}{2} = -1.3 \, J
    \end{align*}
\end{solution}