\item A chain hangs on a thread and touches the surface of a table by its lower end. Show that after the thread has been burned through, the force exerted on the table by the falling part of the chain at any moment is twice as great as the force of pressure exerted by the part already resting on the table.
\begin{solution}
    \begin{center}
        \begin{tikzpicture}
            \pic at (0, 0) {frame=3cm};
        \end{tikzpicture}
    \end{center}
    
    \begin{align*}
        \intertext{1.157 The descending part of the chain is in free fall, it has speed $v=\sqrt{2gy}$ at the instant when the chain descended a distance $y$. The length of the chain which lands on the floor, during the differential time interval $dt$ following this instant is $vdt$.}
        \intertext{For the incoming chain element on the floor, from $d\mathscr{P}_y = F_y dt$ (where $y$-axis is directed down)}
        0 - (\lambda vdt) v &= F_y dt \\
        F_y &= -\lambda v^2 = -2\lambda gy \\
        \intertext{Hence, the force exerted on the falling chain equals $\lambda v^2$ and is directed upward. Therefore from third law the force exerted by the falling chain on the table at the same instant of time becomes $\lambda v^2$ and is directed downward.}
        \intertext{Since a length of chain of weight $(\lambda yg)$ already lies on the table and the table is at rest, the total force on the floor is $(2\lambda yg) + (\lambda yg) = (3 \lambda yg)$ or the weight of a length $3y$ of chain.}
    \end{align*}
\end{solution}
