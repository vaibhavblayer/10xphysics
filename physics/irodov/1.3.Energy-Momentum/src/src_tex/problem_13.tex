\item A body of mass \( m \) is hauled from the Earth's surface by applying a force \( \mathbf{F} \) varying with the height of ascent \( y \) as \( \mathbf{F} = 2 \left(a y - 1\right) mg \), where \( a \) is a positive constant. Find the work performed by this force and the increment of the body's potential energy in the gravitational field of the Earth over the first half of the ascent.
\begin{solution}
    \begin{center}
        \begin{tikzpicture}
            \pic at (0, 0) {frame=3cm};
        \end{tikzpicture}
    \end{center}
    
    % \begin{align*}
    %     \intertext{First, let us find the total height of ascent. At the beginning and at the end of the path, velocity of the body is equal to zero, and therefore the increment of the kinetic energy of the body is also equal to zero. On the other hand, according to work-energy theorem, the increment $\Delta T$ is equal to the algebraic sum of the work $A$ performed by all the forces, i.e. by the force $F$ and gravity, over this path. However, since $\Delta T = 0$ then $A = 0$. Taking into account that the upward direction is assumed to coincide with the positive direction of the $y$-axis, we can write}
    %     A &= \int_0^b (\mathbf{F} + mg) \cdot d\mathbf{r} = \int_0^b (F_y - mg) \, dy\\
    %       &= mg \int_0^b (1 - 2ay) \, dy = mgh (1 - ah) = 0 \quad \text{when } h = \frac{1}{a}
    %     \intertext{The work performed by the force $F$ over the first half of the ascent is}
    %     A_F &= \int_0^{b/2} F_y \, dy = 2mg \int_0^{b/2} (1 - ay) \, dy = \frac{3mg}{4a}
    %     \interttext{The corresponding increment of the potential energy is}
    %     \Delta U &= \frac{mgh}{2} = \frac{mg}{2a}
    % \end{align*}
\end{solution}
