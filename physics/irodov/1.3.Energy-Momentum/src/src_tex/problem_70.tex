\item A disc \( A \) of mass \( m \) sliding over a smooth horizontal surface with velocity \( v \) experiences a perfectly elastic collision with a smooth stationary wall at a point \( O \) (Fig. 1.48). The angle between the motion direction of the disc and the normal of the wall is equal to \( \alpha \). Find:
    \begin{center}
        \begin{tikzpicture}
            \draw[thick] (-0.5,0) -- (0.5,0);
            \draw[thick] (0,0) -- (0,3);
            \node at (-1,3) {A};
            \draw[->] (-1.2,2) -- (0,0);
            \draw (0,-0.2) arc[start angle=-90,end angle=0,radius=0.2];
            \node at (0.2,-0.2) {\(\alpha\)};
            \node at (0,0) {O};
            \draw[->] (0,-1.2) -- (0,-2);
            \node at (0.2,-2) {\(O'\)};
            \draw[thick, ->] (0, 0) -- (1, -1.5)node[pos=1.2]{\(l\)};
        \end{tikzpicture}
    \end{center}
    \begin{enumerate}
        \item The points relative to which the angular momentum \( M \) of the disc remains constant in this process;
        \item The magnitude of the increment of the vector of the disc's angular momentum relative to the point \( O' \) which is located in the plane of the disc's motion at the distance \( l \) from the point \( O \).
    \end{enumerate}\begin{solution}
    \begin{center}
        \begin{tikzpicture}
            \pic at (0, 0) {frame=3cm};
        \end{tikzpicture}
    \end{center}
    
    \begin{align*}
        \intertext{(a) The disk experiences gravity, the force of reaction of the horizontal surface, and the force $R$ of reaction of the wall at the moment of the impact against it. The first two forces counter-balance each other, leaving only the force $\vec{R}$. Its moment relative to any point of the line along which the vector $\vec{R}$ acts or along normal to the wall is equal to zero and therefore the angular momentum of the disk relative to any of these points does not change in the given process.}
        \intertext{(b) During the course of collision with wall the position of disk is same and is equal to $\vec{r_{00'}}$. Obviously the increment in momentum}
        \Delta\vec{M} &= \vec{r_{00'}} \times \Delta\vec{p} = 2mv\! \cos \alpha\, \vec{n}\\
        |\Delta\vec{M}| &= 2mv\! \cos \alpha
        \intertext{Here, $\Delta\vec{M} = 2mv\! \cos \alpha$}
    \end{align*}
\end{solution}