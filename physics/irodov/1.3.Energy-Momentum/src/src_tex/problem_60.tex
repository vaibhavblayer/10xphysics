\item A molecule collides with another, stationary, molecule of the same mass. Demonstrate that the angle of divergence
    \begin{enumerate}
        \item equals 90° when the collision is ideally elastic;
        \item differs from 90° when the collision is inelastic.
    \end{enumerate}\begin{solution}
    \begin{center}
        \begin{tikzpicture}
            \pic at (0, 0) {frame=3cm};
        \end{tikzpicture}
    \end{center}

    \begin{align*}
        \intertext{1.177 (a) Let a molecule come with velocity $\vec{v_1}$ to strike another stationary molecule an just after collision their velocities become $\vec{v_1}'$ and $\vec{v_2}'$, respectively. As the mass of each molecule is same, conservation of linear momentum and conservation of kinetic energy for the system (both molecules), respectively, gives}
        \vec{v_1} &= \vec{v_1}' + \vec{v_2}' \tag{1} \\
        v_1^2 &= v_1'^2 + v_2'^2 \tag{2}
        \intertext{From the property of vector addition it is obvious from the obtained equations that}
        \vec{v_1} \perp \vec{v_2} \text{ or } \vec{v_1} \cdot \vec{v_2} &= 0
        \intertext{(b) Due to the loss of kinetic energy in inelastic collision $v_1^2 > v_1'^2 + v_2'^2$ so, $\vec{v_1} \cdot \vec{v_2} > 0$ and therefore angle of divergence $< 90^\circ$.}
    \end{align*}
\end{solution}