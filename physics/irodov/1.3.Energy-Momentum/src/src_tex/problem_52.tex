\item Particle 1 experiences a perfectly elastic collision with a stationary particle 2. Determine their mass ratio, if
\begin{enumerate}
    \item after a head-on collision the particles fly apart in the opposite directions with equal velocities;
    \item the particles fly apart symmetrically relative to the initial motion direction of particle 1 with the angle of divergence $\theta = 60^\circ$.
\end{enumerate}
\begin{solution}
    \begin{center}
        \begin{tikzpicture}
            \pic at (0, 0) {frame=3cm};
        \end{tikzpicture}
    \end{center}

    \begin{align*}
        \intertext{(a) Being perfectly elastic head on collision, the velocity of $i$-th particle after collision}
        \vec{v}'_i &= 2\vec{v}_C - \vec{v}_i \quad ( \text{where} \ i = 1, \ 2) \ \text{(see previous problem)}\\
        \intertext{So,}
        \vec{v}'_1 &= 2 \left( \dfrac{m_1 \vec{v}_1}{m_1 + m_2} \right) - \vec{v}_1 = \left( \dfrac{m_1 - m_2}{m_1 + m_2} \right) \vec{v}_1 \tag{1}\\
        \vec{v}'_2 &= 2 \left( \dfrac{m_1 \vec{v}_1}{m_1 + m_2} \right) \tag{2}\\
        \intertext{According to the problem}
        \vec{v}'_2 &= -\vec{v}'_1 \\
        \text{so,} \quad 2 \left( \dfrac{m_1 \vec{v}_1}{m_1 + m_2} \right) &= \left( \dfrac{m_2 - m_1}{m_1 + m_2} \right) \vec{v}_1 \ \text{which gives} \ 2m_1 = (m_2- m_1) \\
        \text{Hence,} \quad \dfrac{m_1}{m_2} &= \dfrac{1}{3}
        \intertext{(b) From conservation of linear momentum in the direction perpendicular to initial motion direction of striking particle 1 gives}
        p'_1 \sin 30^\circ &= p'_2 \sin 30^\circ \\
        \text{So,} \quad p'_1 &= p'_2 \tag{1}\\
        \intertext{From conservation of linear momentum}
        \vec{p}_1 &= \vec{p}'_1 + \vec{p}'_2, \quad \vec{p}_1 - \vec{p}'_1 = \vec{p}'_2\\
        \text{so,} \quad p^2_1 + p'^2_1 - 2p_1 p'_1 \cos 30^\circ &= p'^2_2   \tag{2}\\
        \intertext{On using Eq. (1) in Eq. (2) we get on manipulation}
        p'_1 &= \dfrac{p_1}{2 \cos 30^\circ} = \dfrac{p_1}{\sqrt{3}} \tag{3}\\
        \intertext{From conservation of kinetic energy}
        \dfrac{p^2_1}{2m_1} &= \dfrac{p'^2_1}{2m_1} + \dfrac{p'^2_2}{2m_2} \\
        \intertext{On using Eq. (1) and Eq. (3), we get on solving}
        \dfrac{m_1}{m_2} &= 2
    \end{align*}
\end{solution}
