\item A small ball is suspended from a point $O$ by a light thread of length $l$. Then the ball is drawn aside so that the thread deviates through an angle $\theta$ from the vertical and set in motion in a horizontal direction at right angles to the vertical plane in which the thread is located. What is the initial velocity that has to be imparted to the ball so that it could deviate through the maximum angle $\pi/2$ in the process of motion?
\begin{solution}
    \begin{align*}
        \intertext{The swinging sphere experiences two forces: the gravitational force and the tension of the thread. Now, it is clear from the condition, given in the problem, that the moment of these forces about the vertical axis, passing through the point of suspension $N_z = 0$. Consequently, the angular momentum $M_z$ of the sphere relative to the given axis ($z$) is constant.}
        \intertext{Thus,}
        mv_0 \left(l \sin \theta \right) &= mvl \quad \tag{1}
        \intertext{where \( m \) is the mass of sphere and \( v \) is its velocity in the position, when the thread forms an angle \( \pi/2 \) with the vertical. Mechanical energy is also conserved, as the sphere is under the influence of only one other force, i.e. tension, which does not perform any work, as it is always perpendicular to the velocity.}
        \intertext{So,}
        \dfrac{1}{2}mv_0^2 + mgl \cos \theta &= \dfrac{1}{2}mv^2 \quad \tag{2}
        \intertext{From Eqs. (1) and (2), we get}
        v_0 &= \sqrt{2gl/\cos\theta}
    \end{align*}
\end{solution}
