\item A ball moving translationally collides elastically with another, stationary, ball of the same mass. At the moment of impact the angle between the straight line passing through the centres of the balls and the direction of the initial motion of the striking ball is equal to \(\alpha = 45^\circ\). Assuming the balls to be smooth, find the fraction \(\eta\) of the kinetic energy of the striking ball that turned into potential energy at the moment of the maximum deformation.
\begin{solution}
    \begin{center}
        \begin{tikzpicture}
            \pic at (0, 0) {frame=3cm};
        \end{tikzpicture}
    \end{center}
    
    % \begin{align*}
    %     \intertext{At the moment of maximum deformation the velocity of colliding bodies along their common normal $\mathbf{n}$ (here the line joining the mass centres of the balls) must be equal so}
    %     v'_{1n} &= v'_{2n} = v'_{n} \, (\text{say}) \tag{1}
    %     \intertext{As the balls are smooth so, velocity of each ball along their common tangent $\mathbf{t}$ remains constant.}
    %     v'_{1t} &= v_1 = v_1 \sin 45^\circ \text{ and } v'_{2t} = v_{2t} = 0 \tag{2}
    %     \intertext{Now, from conservation of linear momentum}
    %     mu_1 \cos 45^\circ &= 2m v'_{n}, \text{ so, } v'_{n} = \frac{v_1}{2\sqrt{2}} \tag{3}
    %     \intertext{From energy conservation, gain in internal potential energy is due to loss of kinetic energy of the system. Initial kinetic energy of the system is the kinetic energy $T_1$ of striking ball only so, the sought fraction $\eta = 1 - T'_{\text{system}} / T_1$.}
    %     \intertext{But,}
    %     T'_{\text{system}} &= \frac{1}{2} m \left[ v'_{1n}^2 + v'_{1t}^2 \right] + \frac{1}{2} m \left[ v'_{2n}^2 + v'_{2t}^2 \right] \\
    %     \intertext{On using Eqs. (1), (2) and (3), we get}
    %     T'_{\text{system}} &= \frac{1}{2} m \frac{3 v_1^2}{4} \\
    %     T'_{\text{system}} &= \frac{1}{2} m \frac{3 v_1^2}{4} / \frac{1}{2} m v_1^2 = \frac{3}{4}
    %     \intertext{So,}
    %     \frac{T'_{\text{system}}}{T_1} &=  \frac{3}{4}
    %     \intertext{Hence, sought fraction $\eta = 1 - \frac{3}{4} = \frac{1}{4}$.}
    % \end{align*}
\end{solution}
