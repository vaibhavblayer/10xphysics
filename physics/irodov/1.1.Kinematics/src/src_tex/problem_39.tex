
\item A point moves along an arc of a circle of radius \( R \). Its velocity depends on the distance covered \( s \) as \( v = \alpha\sqrt{s} \), where \( \alpha \) is a constant. Find the angle \( \alpha \) between the vector of the total acceleration and the vector of velocity as a function of \( s \).
\begin{solution}
    \begin{center}
        \begin{tikzpicture}
            \pic at (0, 0) {frame=3cm};
        \end{tikzpicture}
    \end{center}
    
    \begin{align*}
        \intertext{From the equation}
        v &= a\sqrt{s}\\
        w_t &= \frac{dv}{dt} = \frac{a}{2\sqrt{s}} \frac{ds}{dt} = \frac{a}{2\sqrt{s}} a\sqrt{s} = \frac{a^2}{2}
        \intertext{and}
        w_n &= \frac{v^2}{R} = \frac{a^2 s}{R}
        \intertext{As \(w_t\) is a positive constant, the speed of the particle increases with time, and the tangential acceleration vector and velocity vector coincides in direction. Hence the angle between \(\mathbf{v}\) and \(\mathbf{w}\) is equal to angle between \(\mathbf{w_t}\) and \(\mathbf{w_n}\) and \(\alpha\) can be found by the formula}
        \tan \alpha &= \left| \frac{w_n}{w_t} \right| = \frac{\frac{a^2 s}{R}}{a^2/2} = \frac{2s}{R}
    \end{align*}
\end{solution}