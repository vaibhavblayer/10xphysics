
\item A rotating disc (Fig. 1.6) moves in the positive direction of the x axis. Find the equation \( y(x) \) describing the position of the instantaneous axis of rotation, if at the initial moment the axis \( C \) of the disc was located at the point \( O \) after which it moved
    \begin{enumerate}
        \item with a constant velocity \( v \), while the disc started rotating counterclockwise with a constant angular acceleration \( \beta \) (the initial angular velocity is equal to zero);
    
        \item with a constant acceleration \( w \) (and the zero initial velocity), while the disc rotates counterclockwise with a constant angular velocity \( \omega \).
    \end{enumerate}

\begin{solution}
    \begin{center}
        \begin{tikzpicture}
            \pic at (0, 0) {frame=3cm};
        \end{tikzpicture}
    \end{center}
    
    \begin{align*}
        \intertext{Therefore the velocity vector of an arbitrary point \( P \) of the solid can be represented as:}
        \vec{v}_P &= \vec{w} \times r_{PI} = \vec{w} \times \rho_{PI} \tag{1}
        \intertext{(where \( \rho_{PI} \) is normal location of point \( P \) relative to instantaneous rotation axis passing through point \( I \).)}
        \intertext{So instantaneous rotation axis  \( I \) is at the perpendicular distance \( \rho_{PI} = v_P / \omega \) from point \( P \).}
        \intertext{On the basis of Eq.~(1) for the centre of mass (C.M.) of the disk, velocity is}
        \vec{v}_C &= \vec{w} \times \rho_{CI} \tag{2}
        \intertext{According to the problem \( \vec{v}_C\ \shortparallel\ \hat{i} \) and \( \vec{w}\ \shortparallel\ \hat{k} \), so to satisfy the Eq.~(2), \( \rho_{CI} \) is directed towards \((- \hat{j})\). Hence point \( I \) is at a distance \( \rho_{CI} = y \), above the centre of the disk along y-axis. Using all these facts in Eq.~(2), we get}
        v_C &= \omega y \quad \text{or} \quad y = \dfrac{v_C}{\omega} \tag{3}
        \\
        \intertext{(a) From the angular kinematical equation}
        \omega_z &= \omega_{0z} + \beta_z t \tag{4} \\
        \omega &= \beta t \\
        \intertext{On the other hand}
        x &= vt, \quad t = x/v \quad \text{(where \( x \) is the \( x \) coordinate of the C.M.)} \tag{5}
        \intertext{From Eqs.~(4) and (5),}
        \omega &= \dfrac{\beta x}{v}
        \intertext{Using this value of \( \omega \) in Eq.~(3) we get}
        y &= \dfrac{v_C}{\omega} = \dfrac{v}{\beta x/v} = \dfrac{v^2}{\beta x} \quad \text{(hyperbola)} \\
        \intertext{(b) As centre \(C\) moves with constant acceleration \( w\), with zero initial velocity}
        x &= \dfrac{1}{2} wt^2 \quad \text{and} \quad v_C = wt \\
        \intertext{Therefore,}
        v_C &= w\sqrt{\dfrac{2x}{w}} = \sqrt{2 x w} \\
        \intertext{Hence,}
        y &= \dfrac{v_C}{\omega} = \dfrac{\sqrt{2 w x}}{\omega}\quad \text{(parabola)}
    \end{align*}
\end{solution}
