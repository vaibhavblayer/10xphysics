\documentclass[12 pt]{book}
\usepackage{amsmath}
\usepackage{amsthm}
\usepackage[paperwidth=8.27 in,paperheight=11.69 in,left=25 mm, right=25 mm, top=20 mm, bottom=25 mm]{geometry}
\usepackage{graphics}
\usepackage{fontawesome}
\usepackage{enumitem}
\usepackage{marvosym}
\newcommand{\myitem}{\refstepcounter{enumi}\item[$^\star$\theenumi.]}
\newcommand{\mmyitem}{\refstepcounter{enumi}\item[$^{\star \star}$\theenumi.]}
\setcounter{page}{401}

\usepackage[utf8]{inputenc}
\usepackage{xcolor}
\setlength{\arrayrulewidth}{0.1 mm}
\definecolor{Mycolor2}{HTML}{33cccc}


\usepackage{fancyhdr}

\pagestyle{fancy}
\fancyhf{}
\setlength{\headheight}{10 mm}

\renewcommand{\headrulewidth}{0 mm}


%%----FONT &&& MATHS_FONT----%%

\usepackage{amssymb}
\usepackage{upgreek,xspace}
\newcommand*{\rom}[1]{\expandafter\@\romannumeral #1}

\usepackage[utopia]{mathdesign}
% \renewcommand{\familydefault}{\sfdefault}
%\usepackage{newtxtext,newtxmath}
% \usepackage[scaled=1]{helvet}
\newcommand*\Times{\fontfamily{ptm}\selectfont}

%%%------PACAKAGES------%%%

\usepackage[letterspace=200]{microtype}
\usepackage{enumitem}
\usepackage{multicol}
\usepackage{pgfplots}
\pgfplotsset{width=8cm,compat=1.16}
\usepackage{tikz}
\usepgfplotslibrary{fillbetween}
\usetikzlibrary{quotes,angles,patterns,through,calc}
\usepgflibrary{arrows.meta}
\usetikzlibrary{optics}
\usetikzlibrary{intersections}

\usepackage[version=4]{mhchem}
\usepackage{chemformula}
\usepackage{elements}

%\DeclareMathSymbol{\shortminus}{\mathbin}{AMSa}{"39}

\usepackage{mathtools}
\usepackage{old-arrows}
\usepackage[b]{esvect}

\newcommand{\midarrow}{\tikz \draw[-Stealth] (0,0) -- +(.1,0);}
\usetikzlibrary{mindmap}
\usetikzlibrary{scopes}
\usetikzlibrary{backgrounds}
\pgfsetlayers{background,main,foreground}
\pgfdeclarelayer{background}
\pgfdeclarelayer{foreground}
\usetikzlibrary{trees}
\usetikzlibrary{shadings}
%\tikzset{every node/.append style = {draw=black,thin}}
\usetikzlibrary{shadows}



\usepackage{color}
\usepackage[autostyle]{csquotes}
\usepackage{xcolor}
\definecolor{Mycolor2}{HTML}{33cccc}
\definecolor{One}{HTML}{336666}
\definecolor{Two}{HTML}{666666}
\definecolor{Three}{HTML}{cc6699}
\definecolor{tomato}{HTML}{FF6347}
\definecolor{darkblue}{HTML}{2c3e50}




\newcommand{\E}{{\Times{\Huge{\textit{\textcolor{black!65}{E{\textcolor{red!95}{.}}}}}}}}


\newcommand*\circled[1]{\tikz[thin, baseline=(char.base)]{
            \node[shape=circle,draw,inner sep=1pt] (char) {#1};}}
      

\newenvironment{ctikz}{\begin{center}\begin{tikzpicture}} {\end{tikzpicture} \end{center}}

\newcommand{\physics}{\normalsize{\textcolor{tomato}{\textls*[100]{{\hspace*{75 mm} @10xphysics}}}}}


\newenvironment{my-title}
{
	\begin{center}
	\begin{itshape}
	\Large\Times\textit{}
}
{
	\end{itshape}
	\end{center}
}


\newenvironment{definition}
{
	\begin{center}
	\begin{itshape}
	\large\Times\textit{}
}
{
	\end{itshape}
	\end{center}
}


\newenvironment{note}
{
	\begin{center}
	\begin{itshape}
	\small\Times\textit{}
}
{
	\end{itshape}
	\end{center}
}

\newenvironment{chemistry}
{
	\begin{center}
	\begin{itshape}
	\Large\Times\textit{}
}
{
	\end{itshape}
	\end{center}
}

\title{Nuclei Decay}

\date{}

\begin{document}

% \nopagecolor
%\boldmath
%\color{white!100}
% \LARGE

%\pagecolor{Mycolor2!80}
%\pagecolor{black!35}
% \setlength{\parindent}{0pt}

\begin{center}
	\textsc{Nuclei Decay}\\[10mm]
\end{center}


\begin{my-title}
alpha decay
\end{my-title}

\begin{definition}
An alpha particle is a helium nucleus. Thus a nucleus emitting an alpha particle loses two protons and two neutrons.
\end{definition}


\begin{chemistry}
\ch{$_ZX^A$ \quad ->[$\alpha$-decay] \quad $_{Z-2}Y^{A-4}$ \quad + \quad $_2$He$^4$} \\[7.5 mm]
\ch{$_{92}U^{238}$ \quad ->[$\alpha$-decay] \quad $_{90}$Th$^{234}$ \quad + \quad $_2$He$^4$}
\end{chemistry}

\vspace*{10mm}


\begin{my-title}
Beta decay
\end{my-title}

\begin{definition}
Beta decay involves the emission of electrons or positron. A positron is a form of antimatter, which has a charge equal to $+e$ and a mass equal to that of electron.
\end{definition}


\begin{note}
\begin{tikzpicture}
[edge from parent fork down, sibling distance=40mm, level distance=25mm,
every node/.style={draw,black,line width=0.75,rounded corners},
edge from parent/.style={black,line width=0.75,draw,line cap=round}] 
\node {Beta Decay}
      child {node {$\upbeta^-$ decay}}
      child {node {$\upbeta^+$ decay}}
      child {node {electron capture}};
\end{tikzpicture}
\end{note}


\vspace*{10mm}

\begin{my-title}
$\upbeta^-$ decay
\end{my-title}

\begin{definition}
In $\upbeta^-$ decay, a neutron in the nucleus is transformed into a proton, an electron and an anti-neutrino ($\overline{\upnu}$). It has zero rest mass, is chargeless.
\end{definition}


\begin{chemistry}
\ce{n  $~~$ -> $~~$ p $~~$ + $~~$ e- $~~$ + $~~$ $\overline{\upnu}$}\\[8 mm]
\ce{$_ZX^A$ $~~~$ ->[$\upbeta^-$] $~~~$ $_{Z+1}Y^A~~$ +$~~$ e- $~~$ + $~~$ $\overline{\upnu}$}\\[5 mm]
\ce{$_6C^{14}$ $~~$ ->[$\upbeta^-$] $~~$ $_7N^{14}~~$ + $~~$ e- $~~$ + $~~$ $\overline{\upnu}$ }
\end{chemistry}

\pagebreak

\begin{my-title}
$\upbeta^+$ decay
\end{my-title}

\begin{definition}
In $\upbeta^+$ decay, a proton changes into a neutron with the emission of a positron and a neutrino ($\upnu$).
\end{definition}


\begin{chemistry}
\ce{p  $~~$ -> $~~$ n $~~$ + $~~$ e+ $~~$ + $~~$ $\upnu$}\\[8 mm]
\ce{$_ZX^A$ $~~~$ ->[$\upbeta^+$] $~~~$ $_{Z-1}Y^A~~$ +$~~$ e+ $~~$ + $~~$ $\upnu$}\\[5 mm]
\ce{$_7N^{13}$ $~~$ ->[$\upbeta^+$] $~~$ $_6C^{13}~~$ + $~~$ e+ $~~$ + $~~$ $\upnu$ }
\end{chemistry}



\vspace*{10mm}

\begin{my-title}
electron capture
\end{my-title}

\begin{definition}
This occurs when a parent nucleus captures one of its own orbital electrons and emits a neutrino.
\end{definition}


\begin{chemistry}
\ce{$_ZX^A$ $~~~$  +$~~$ e- $~~$ ->[{\small{K-capture}}] $~~~$ $_{Z-1}Y^A~~$  + $~~$ $\upnu$}\\[7 mm]
\ce{$_4$Be$^{7}$ $~~$ +$~~$ e- $~~$ ->[{\small{K-capture}}] $~~$ $_3$Li$^{7}~~$  + $~~$ $\upnu$ }
\end{chemistry}

\begin{note}
{In most cases, it is K-shell electron that is captured, for this reason the process is referred to as K-capture.}
\end{note}

\pagebreak

     



\end{document}
