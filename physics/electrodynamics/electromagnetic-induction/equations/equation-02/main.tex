\documentclass{article}
\usepackage{v-equation}
\vgeometry
\newcommandx*\varea[5][1=title, 2=0,3=0,4=$d\vec{S}$, 5=$\vec{B}$, usedefault=@]{
	%\coordinate (O) at (0, 0);
	\tzcoors(3,1.5)(A)(5,2)(B)(5.5,3)(C)(4.5,4)(D)(2.5,3)(E);
	%\tzdots*<#2, #3>(A)(B)(C)(D)(E);
	\tztos[#1]<#2, #3>
		(A)[out=-30,in=210]
		(B)[out=30,in=-90]
		(C)[out=90,in=-30]
		(D)[out=150,in=80]
		(E)[out=-90,in=150]
		(A);
	\coordinate (O) at ($0.5*(A)+0.5*(D)$);
	\draw[pattern=crosshatch] (O) --++(1, 0)--++(0.3, 0.4)coordinate (X)--++(-1, 0)--cycle;
	\coordinate (c) at ($0.5*(O)+0.5*(X)$);
	\coordinate (a) at ($0.5*(O)+0.5*(X)+(1, 1.3)$);
	\coordinate (b) at ($0.5*(O)+0.5*(X)+(0, 1.5)$);
	\tzline+[->](c)(0, 1.5){#4}[a]
	\tzline+[->](c)(1, 1.3){#5}[ar]
	\tzanglemark(a)(c)(b){$\theta$}
}
\begin{document}
\vtitle[Faraday’s laws of electromagnetic induction]
\begin{center}
\begin{tikzpicture}
[thick]
	\begin{scope}[scale=1.3]
	\varea[pattern=dots]
	\end{scope}
\end{tikzpicture}
\end{center}
\vspace*{\fill}
\begin{itemize}
\item Faraday’s first law
\begin{itemize}
\item Whenever the amount of magnetic flux linked with a circuit changes, an emf is induced in the circuit.
The actual number of magnetic lines passing through the circuit does not matter to the value of the induced emf. Induced emf is determined by the rate at which the magnetic flux changes.
\end{itemize}

\pagebreak
\item Faraday’s second law
\begin{itemize}
\item The magnitude of the induced emf in a circuit is equal to the rate of change of magnetic flux through the circuit.
\end{itemize}

\end{itemize}
\begin{align*}
\Aboxed{e = - \dfrac{d\!\phi_B}{d\!t}}
\end{align*}
\vspace*{\fill}
\begin{center}
\begin{tikzpicture}
	\foreach \x in {0, 1, ..., 6}{
		\foreach \y in {0, 1, ..., 4}{
		\def\len{0.07}
			\tzline+(\x, \y)(\len, \len)
			\tzline+(\x, \y)(-\len, -\len)
			\tzline+(\x, \y)(\len, -\len)
			\tzline+(\x, \y)(-\len, \len)
		}
	};
	\begin{scope}[yshift=1.25cm, xshift=-2cm, thick]
		\draw (0, 0) rectangle (4, 1.5);
		\fill[pattern=north east lines] (2, 0) rectangle (4, 1.5);
		\draw[->] (4, 0.75)--++(1.5, 0);
	\end{scope}
\end{tikzpicture}
\end{center}
\vspace*{\fill}
\pagebreak
\vspace*{\fill}
\begin{center}
	\fbox{\qrcode[height=2cm]{https://drive.google.com/drive/folders/1LfdkltaMoM28zqjU2kcviD55AGNOwHyV?usp=share_link}}
\end{center}
\vspace*{\fill}

\end{document}
