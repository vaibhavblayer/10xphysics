\documentclass{article}
\usepackage{v-equation}
\vgeometry

\begin{document}

\def\gdrive{https://drive.google.com/drive/folders/134Dp6Lf0MMG1RS7dcQl6RoIlbQ4pKlSL?usp=share_link}

\vtitle[Magnetic Force : Direction]
\begin{center}
\begin{tikzpicture}
[xscale=1.2, yscale=1.2, very thick, >=latex]
\draw[->] (0,0,0)--(2,0,0) node[below]{$\vec{v}$};
\draw[->] (0,0,0)--(0,1.5,0) node[left]{$\vec{F}$};
\draw[->] (0,0,0)--(1,0,-2) node[above]{$\vec{B}$};
\draw (0.5,0,0) arc[start angle=0, end angle=24, radius=0.5];
\draw [line width=0 mm, opacity=0.35,pattern=grid, pattern color=black!50] (0,0,0)--(2,0,0)--(3,0,-2)--(1,0,-2)--(0,0,0);
\end{tikzpicture}
\end{center}
Direction of magnetic force will be in the direction of the plane perpendicular to both $\vec{v}$ and $\vec{B}$, because it is defined as cross product.
\begin{itemize}
\item It can be found using right-hand thumb rule.
\item Just point your fingers in the direction of $\vec{v}$ such that they can be curled into $\vec{B}$, the thumb points in the direction of $\vec{F}$. 
\end{itemize}
\vspace*{\fill}
\begin{align*}
\vec{F}=q_0 \left( \vec{v} \times \vec{B} \right)
\end{align*}
\vspace*{\fill}

\pagebreak

\vspace*{\fill}
\begin{center}
    \fbox{\qrcode[height=2cm]{\gdrive}}
\end{center}
\vspace*{\fill}
\end{document}
