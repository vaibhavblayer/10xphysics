

\section*{Differentiability of a Function}

As defined earlier,

\begin{align*}
    f'(x) &= \lim_{\Delta x \to 0} \frac{f(x + \Delta x) - f(x)}{\Delta x}
\end{align*}

The limit may or may not exist. Hence, a function \( f(x) \) is said to be differentiable at a point \( a \), if 

\begin{align*}
    \lim_{h \to 0} \frac{f(a + h) - f(a)}{h} &= \lim_{h \to +0} \frac{f(a + h) - f(a)}{h} = \text{some finite real number}.
\end{align*}

Left hand limit and right hand limit are called left hand derivative and right hand derivative respectively at \( a \). Thus, a function \( f(x) \) is differentiable at a point \( x = a \), if left hand derivative and right hand derivative at \( a \) are equal.


