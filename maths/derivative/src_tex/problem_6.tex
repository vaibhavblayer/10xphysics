


\textbf{Example.} Consider the function \( f(x) = 
   \begin{cases} 
   x^2, & x \geq 0 \\
   -x, & x < 0 
   \end{cases} \)

Left hand derivative at \( x = 0 \)

\begin{align*}
    &= \lim_{h \to 0 - 0} \frac{f(0 + h) - f(0)}{h} \\
    &= \lim_{h \to 0 - 0} \frac{-h - 0}{h} \\
    &= -1
\end{align*}

Right hand derivative at \( x = 0 \)

\begin{align*}
    &= \lim_{h \to 0 + 0} \frac{f(0 + h) - f(0)}{h} \\
    &= \lim_{h \to 0 + 0} \frac{h^2 - 0}{h} \\
    &= \lim_{h \to 0 + 0} h \\
    &= 0
\end{align*}

Left hand derivative \(\neq\) Right hand derivative at \( x = 0 \). \\
Hence, \( f(x) \) is not differentiable at \( x = 0 \). \\
A function is said to be a differentiable function if it is differentiable at all points in its domain of definition.

\textbf{Continuity and Differentiability}

Suppose a function \( f(x) \) is differentiable at \( x = a \) i.e., \( f'(a) \) exists.

\begin{align*}
    \text{Now,} \quad \lim_{h \to 0} [f(a + h) - f(a)] &= \lim_{h \to 0} \frac{f(a + h) - f(a)}{h} \cdot h \\
    &= \lim_{h \to 0} \frac{f(a + h) - f(a)}{h} \cdot \lim_{h \to 0} h \\
    &= f'(a) \cdot 0 = 0 \\
    \therefore \lim_{h \to 0} [f(a + h) - f(a)] &= 0
\intertext{or}
    \lim_{h \to 0} f(a + h) &= f(a)
\end{align*}

Thus, \( f(x) \) is continuous at \( x = a \). \\
Hence, if a function \( f(x) \) is differentiable at some point \( x = a \), then it is continuous at that point. But, the converse need not be true.

\textbf{Higher Order Derivatives}

\( f'(x) \), as defined earlier, is called the first order derivative of \( f(x) \) with respect to \( x \). Derivative of \( f'(x) \) with respect to \( x \), denoted by \( f''(x) \), is called the second order derivative of \( f(x) \) with respect to \( x \). Similarly, \( n \)th order derivative of \( f(x) \), denoted by \( f^n(x) \), is defined.

\textbf{Let} \\
\( y = f(x) \)

We know that
\begin{equation}
    \frac{dy}{dx} = f'(x)
\end{equation}
