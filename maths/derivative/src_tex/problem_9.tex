


\begin{align*}
& \text{(xi) } \frac{d(\sin^{-1} x)}{dx} = \frac{1}{\sqrt{1 - x^2}} \\
& \text{(xii) } \frac{d(\cos^{-1} x)}{dx} = \frac{-1}{\sqrt{1 - x^2}} \\
& \text{(xiii) } \frac{d(\tan^{-1} x)}{dx} = \frac{1}{1 + x^2}
\end{align*}

\noindent
5. A function \( f(x) \) is said to be differentiable at a point \( a \), if
\begin{align*}
\lim_{h \to 0^-} \frac{f(a + h) - f(a)}{h} &= \lim_{h \to 0^+} \frac{f(a + h) - f(a)}{h} = \text{some finite real number}.
\end{align*}

\noindent
6. If a function \( f(x) \) is differentiable at a point \( x = a \), then, it is continuous at that point. But, the converse need not be true.

\noindent
7. Let \( y = f(x) \). \( \frac{dy}{dx} \) is called first order derivative of \( y \) with respect to \( x \).

Derivative of \( \frac{dy}{dx} \) with respect to \( x \), denoted by \( \frac{d^2 y}{dx^2} \), is called second order derivative of \( y \) with respect to \( x \). Similarly, other higher order derivatives are defined.

\noindent
8. For differentiable functions \( u, v \) of \( x \) and constant \( c \):
\begin{align*}
(a) \ \frac{d(cu)}{dx} &= c \frac{du}{dx} \\
(b) \ \frac{d(u+v)}{dx} &= \frac{du}{dx} + \frac{dv}{dx} \\
(c) \ \frac{d(u-v)}{dx} &= \frac{du}{dx} - \frac{dv}{dx} \\
(d) \ \frac{d(uv)}{dx} &= u \frac{dv}{dx} + v \frac{du}{dx} \\
(e) \ \frac{d(u/v)}{dx} &= \frac{v \frac{du}{dx} - u \frac{dv}{dx}}{v^2} \quad \text{provided } v \ne 0
\end{align*}

\noindent
9. Let \( y = f(x) \) and \( x = g(t) \) be differentiable functions of \( x \) and \( t \) respectively, then,
\begin{align*}
\frac{dy}{dt} = \frac{dy}{dx} \cdot \frac{dx}{dt} = f'(x) \cdot g'(t)
\end{align*}

