
% \subsection*{In a Snapshot}
% \begin{itemize}
%     \item Differential Coefficient (or Derivative)
%     \item Differential Coefficients of Some Functions
%     \item Differentiability of a Function
%     \item Continuity and Differentiability
%     \item Higher Order Derivatives
%     \item Some Rules of Differentiation
% \end{itemize}

Study of motion is vital to Physics. The description of motion is one fundamental aspect of the study of motion. The techniques of differentiation and integration make the description of motion precise. Let us discuss an example to understand how these techniques make the description of motion precise. Suppose a car is moving on the road. If the car is moving uniformly, average speed (defined as total distance travelled divided by total time elapsed) is precise measure of how fast the car is moving. But, generally a car moves with varying speeds (\textit{i.e.}, the motion is not uniform). So, average speed during an interval gives only a rough idea about how fast the car is moving.

\begin{align*}
    \intertext{Let $S(t)$ = distance travelled by the car during time interval $0$ to $t$.}
    \intertext{$\therefore \quad$ Distance travelled during the interval $t$ to $t + \Delta t$}
    &= S(t + \Delta t) - S(t)
    \intertext{Let} 
    \Delta S &= S(t + \Delta t) - S(t)
    \intertext{Average speed during time-interval $t$ to $t + \Delta t$}
    &= \frac{\Delta S}{\Delta t}
    \intertext{As $\Delta t \to 0$, $\frac{\Delta S}{\Delta t}$ tends to be a measure of how fast the car is moving at time $t$. Hence, instantaneous speed at time $t$ is defined as}
    v(t) &= \lim_{\Delta t \to 0} \frac{\Delta S}{\Delta t} \\
    &= \lim_{\Delta t \to 0} \frac{S(t + \Delta t) - S(t)}{\Delta t}
    \intertext{$\lim_{\Delta t \to 0} \frac{\Delta S}{\Delta t}$, denoted by $\frac{dS}{dt}$, is called the differential coefficient of $S(t)$ with respect to $t$. $v(t)$ is precise measure of how fast the car is moving at time $t$.}
\end{align*}



Now, we can talk of the motion of the car at a particular instant of time rather than over a time interval.

Thus, the technique of differentiation makes the description of motion precise. But, this is not all about differentiation. The techniques of differentiation and integration have extensive applications in Science, in general and Physics, in particular.