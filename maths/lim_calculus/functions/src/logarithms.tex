\addcontentsline{toc}{subsection}{Logarithmic Functions}
\vbtitle{Logarithmic Functions}
\begin{align*}
    8 &= 2^3\\
    \intertext{In the equation above, we know that \( 2^3 = 8 \), but what if we want to find the exponent to which 2 must be raised to obtain 8? This is where logarithms come into play.}
\end{align*}
\begin{itemize}
    \item \textbf{Exponents:} find the final result of growth
        \vbdefinition{
            Exponents are the operations that show how many times a number is multiplied by itself. They help us find the final result of growth.
            \begin{align*}
                2^3 &= 8
            \end{align*}
        }
    \item \textbf{Logarithms:} reveal the inputs that caused the growth
        \vbdefinition{
            Logarithms are the inverse operations of exponentiation. They help us find the exponent to which a given base must be raised to obtain a specific number.
            \begin{align*}
                \log_2 8 &= 3\\
                \intertext{In the equation above, the logarithm of 8 to the base 2 is 3, which means that 2 must be raised to the power of 3 to get 8.}
            \end{align*}
            \begin{itemize}
                \item[\textsc{\textbf{Argument:}}] the number whose logarithm is being found, in the above example, 8 is the argument.\\
                Is it possible to find the logarithm of a negative number? Let's explore this question.
                    \begin{align*}
                        \log_2 (-8) &= \text{?}
                        \intertext{Above equation tell us that to get -8, to what power 2 must be raised.}
                        -8 &= 2^x
                        \intertext{There is no real number \( x \) that satisfies the equation above. Therefore, the logarithm of a negative number is undefined.}
                    \end{align*}
                \item[\textsc{\textbf{Base:}}] the number that is raised to the power of the logarithm, in the above example, 2 is the base.\\
                Is it possible to find the logarithm of a number to the base 1? Let's explore this question.
                    \begin{align*}
                        \log_1 8 &= \text{?}
                        \intertext{Above equation tell us that to get 8, to what power 1 must be raised.}
                        8 &= 1^x
                        \intertext{The equation above is true for any real number \( x \). Therefore, the logarithm of a number to the base 1 is undefined.}
                    \end{align*}
                \item[\textsc{\textbf{Negative Base:}}] Is it possible to find the logarithm of a number with a negative base? Let's explore this question.
                    \begin{align*}
                        \log_{-2} 8 &= \text{?}
                        \intertext{The equation above asks to what power -2 must be raised to get 8.}
                        8 &= (-2)^x
                        \intertext{Let's consider different values of \( x \):}
                        (-2)^1 &= -2\\
                        (-2)^2 &= 4\\
                        (-2)^3 &= -8\\
                        (-2)^4 &= 16
                        \intertext{We observe that \( (-2)^x \) can only result in positive values when \( x \) is an even integer and negative values when \( x \) is an odd integer. There is no real \( x \) that results in 8 being always positive. Therefore, the logarithm with a negative base is not defined.}
                    \end{align*}
            \end{itemize}
            Now, we can conclude that the logarithm of a negative number and the logarithm to the base 1 or a negative base are undefined.\\[2mm]
            \textbf{Most of the time, we will deal with logarithms with base 10 and \( e \):}
            \begin{itemize}
                \item[\textsc{\textbf{Base 10:}}] The logarithm with base 10 is called the common logarithm and is denoted as \( \log_{10}(x) \) or simply \( \log(x) \). For example, \( \log_{10}(100) = 2 \) because \( 10^2 = 100 \).
                \item[\textsc{\textbf{Base \( e \):}}] The logarithm with base \( e \) is called the natural logarithm and is denoted as \( \ln(x) \). For example, \( \ln(e^2) = 2 \) because \( e^2 = e^2 \).
            \end{itemize}
        }
\end{itemize}
    \vbdefinition{
        The logarithm of a number \( x \) to the base \( b \) is the exponent to which the base \( b \) must be raised to obtain the number \( x \). The logarithm is denoted by \( \log_b x \), read as "log of \( x \) to the base \( b \)".
    }

    \[ f(x)=\log_b x \qquad \textit{where } x,b > 0 \textit{ and } b \neq 1 \]

    \vspace*{10mm}

\addcontentsline{toc}{subsubsection}{Properties of Logarithms}
\vbsubtitle{Properties of Logarithms:}
\begin{enumerate}
    \item Log to exponential
        \begin{align*}
            \log_b a &= c \\
            \Rightarrow a &= b^c
        \end{align*}
    
    \item $\log_b x^n = n\log_b x$ 
        \begin{align*}
            \log_b x^n &= t_L \tag{LHS}\\
            \Rightarrow x^n &= b^{t_L} \tag{1}\\ 
            n\log_b x &= t_R \tag{RHS}\\
            \log_b x &= \frac{t_R}{n}\\
            x &= b^{t_R/n}\\
            \intertext{Raise both sides to the power of \( n \)}
            \Rightarrow x^n &= b^{t_R} \tag{2}\\
            \intertext{From the above two equations, we can say that}
            t_L &= t_R\\
            \Aboxed{\log_b x^n &= n\log_b x}
        \end{align*}

    \item $\log_b a = \frac{\log_k a}{\log_k b}$
        \begin{align*}
            \log_b(a) &= c
            \intertext{This implies:}
            b^c &= a
            \intertext{Take the logarithm of both sides with respect to a new base \( k \):}
            \log_k(b^c) &= \log_k(a)
            \intertext{Using the power rule of logarithms \( \log_k(b^c) = c \cdot \log_k(b) \), we get:}
            c \cdot \log_k(b) &= \log_k(a)
            \intertext{Since \( c = \log_b(a) \), substitute \( c \) back into the equation:}
            \log_b(a) \cdot \log_k(b) &= \log_k(a)
            \intertext{Divide both sides by \( \log_k(b) \):}
            \Aboxed{\log_b(a) &= \frac{\log_k(a)}{\log_k(b)}}
        \end{align*}
    

    \item $\log_{a^m} x = \frac{1}{m} \log_a x$
        \begin{align*}
            \log_{a^m} x &= t_L \tag{LHS}\\
            \Rightarrow x &= a^{mt_L} \tag{1}\\
            \frac{1}{m}\log_a x &= t_R \tag{RHS}\\
            \log_a x &= mt_R\\
            x &= a^{mt_R}\\
            \intertext{Raise both sides to the power of \( m \)}
            \Rightarrow x &= a^{mt_R} \tag{2}\\
            \intertext{From the above two equations, we can say that}
            t_L &= t_R\\
            \Aboxed{\log_{a^m} x &= \frac{1}{m} \log_a x}
        \end{align*}

    \item $\log_{b^m} x^n = \left(\frac{n}{m}\right)\log_b x$ \hfill \textit{Combine rule 2 and 4.}

    \item $\log_b (xy) = \log_b x + \log_b y$
        \begin{align*}
            \intertext{From RHS:}
            \log_b x &= m \\ 
            x &= b^m \tag{1}\\
            \log_b y &= n\\
            y &= b^n \tag{2}\\
            \intertext{Multiply the equations (1) and (2)} 
            xy &= b^{m+n} \tag{3}\\
            \intertext{Take the logarithm of both sides with respect to base \( b \)}
            \log_b(xy) &= \log_b b^{m+n}\\
            \log_b(xy) &= m+n\\
            \Aboxed{\log_b(xy) &= \log_b x + \log_b y}
        \end{align*}

    \item $\log_b \left(\frac{x}{y}\right) = \log_b x - \log_b y$
        \begin{align*}
            \intertext{From RHS:}
            \log_b x &= m \\ 
            x &= b^m \tag{1}\\
            \log_b y &= n\\
            y &= b^n \tag{2}\\
            \intertext{Divide the equations (1) and (2)} 
            \frac{x}{y} &= b^{m-n} \tag{3}\\
            \intertext{Take the logarithm of both sides with respect to base \( b \)}
            \log_b\left(\frac{x}{y}\right) &= \log_b b^{m-n}\\
            \log_b\left(\frac{x}{y}\right) &= m-n\\
            \Aboxed{\log_b\left(\frac{x}{y}\right) &= \log_b x - \log_b y}
        \end{align*}

    \item $b^{\log_b x} = x$
        \begin{align*}
            y &= \log_b(x)
            \intertext{This implies by the definition of logarithms:}
            b^y &= x
            \intertext{Since \( y = \log_b(x) \), we substitute \( y \) back into the equation:}
            \Aboxed{b^{\log_b(x)} &= x}
        \end{align*}

    \item $\ln x = \ln(10) \log_{10}x$
        \begin{align*}
            \ln x &= \log_e x\\
            &= \frac{\log_{10}x}{\log_{10}e}\\
            &= \frac{\log_{10}x}{\frac{\log_e e}{\log_e {10}}}\\
            &= \frac{\log_{10}x}{\frac{1}{\ln 10}}\\
           \Aboxed{\ln x &= \ln (10) \log_{10}x}\\
           \intertext{We can approximate the value of \( \ln (10) \) as 2.3, $e=2.718, \quad e^{2.3} \approx 10$,}
           \Aboxed{\ln x &\approx 2.3 \log_{10}x}
        \end{align*}
\end{enumerate}


