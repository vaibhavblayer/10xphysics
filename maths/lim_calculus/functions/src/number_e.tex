\addcontentsline{toc}{subsection}{The Number \( e \)}
\vbtitle{The Number \( e \)}
\setlength{\jot}{10pt}
\begin{align*}
    \left(1 + \frac{1}{1}\right)^1 &= 2\\
    \left(1 + \frac{1}{2}\right)^2 &= 2.25\\
    \left(1 + \frac{1}{10}\right)^{10} &= 2.59374\\
    \left(1 + \frac{1}{100}\right)^{100} &= 2.70481\\
    \left(1 + \frac{1}{1000}\right)^{1000} &= 2.71692\\
    \left(1 + \frac{1}{10000}\right)^{10000} &= 2.71814\\
    \left(1 + \frac{1}{100000}\right)^{100000} &= 2.71826\\
    \left(1 + \frac{1}{1000000}\right)^{1000000} &= 2.71828\\[5mm]
    \intertext{As you can see from the pattern above, as the value of \( n \) increases, the expression \( \left(1 + \frac{1}{n}\right)^n \) gets closer and closer to a particular number. This number is denoted by \( e \), which is approximately equal to 2.71828.}
    \intertext{The value of \( \left(1 + \frac{1}{n}\right)^n \) changes rapidly at first but then starts to stabilize as \( n \) becomes larger. The difference between successive values becomes smaller and smaller, indicating that the sequence is converging.}
    \intertext{This process of the value approaching \( e \) shows that for very large \( n \), the value of \( \left(1 + \frac{1}{n}\right)^n \) approaches \( e \), but never exceeds 3. This special number \( e \) is a fundamental constant in mathematics, especially in calculus.}
    \Aboxed{\lim_{n \to \infty} \left(1 + \frac{1}{n}\right)^n &= e}
\end{align*}