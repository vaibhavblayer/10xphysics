
\documentclass{article}
\usepackage{amsmath}

\begin{document}

158 \hspace{1mm} | \hspace{1mm} Mathematics For Learning Physics

\begin{align*}
(ii) & \int_a^b f(x) - g(x) \, dx = \int_a^b f(x) \, dx - \int_a^b g(x) \, dx \\
(iii) & \int_a^b c f(x) \, dx = c \int_a^b f(x) \, dx, \text{ where } c \text{ is a constant.} \\
(iv) & \int_a^b f(x) \, dx = \int_a^c f(x) \, dx + \int_c^b f(x) \, dx \\
(v) & \int_a^c f(x) \, dx = c \int_a^c f(x) \, dx, \text{ where } c \text{ is a constant.}
\end{align*}

\hspace{3mm} Some Important Indefinite Integrals

\begin{align*}
1. & \int x^n dx = \frac{x^{n+1}}{n+1} + c, n \ne -1, \, n \text{ is a rational number and } c \text{ is a constant.} \\
& \intertext{Take derivative:}
& \frac{d}{dx} \left( \frac{x^{n+1}}{n+1} + c \right) = \frac{d}{dx} \left( \frac{x^{n+1}}{n+1} \right) + 0 = (n+1) \cdot \frac{x^{n+1-1}}{n+1} = x^n \\
& \therefore \int x^n dx = \frac{x^{n+1}}{n+1} + c
\end{align*}

\begin{align*}
2. & \int \frac{1}{x} \, dx = \ln x + c \\
& \intertext{Take derivative:}
& \frac{d}{dx} \left( \ln x + c \right) = \frac{1}{x}
\end{align*}
\begin{align*}
& \therefore \int \frac{1}{x} \, dx = \ln x + c
\end{align*}

\intertext{Similarly, the following results can be proved.}
3. & \int \sin x \, dx = -\cos x + c \\
4. & \int \cos x \, dx = \sin x + c \\
5. & \int \tan x \, dx = \ln (\sec x) + c \\
6. & \int \cot x \, dx = \ln (\sin x) + c \\
7. & \int \sec x \, dx = \ln (\sec x + \tan x) + c \\
8. & \int \csc x \, dx = \ln (\csc x - \cot x) + c \\
9. & \int e^x \, dx = e^x + c
\end{align*}

\textbf{Techniques of Integration}

1. Integration by Substitution

(i) \(\int (x + 1)^2 dx\)

Let \( y = x + 1 \)

\end{document}
