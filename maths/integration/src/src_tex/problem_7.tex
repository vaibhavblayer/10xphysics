
\documentclass{article}
\usepackage{amsmath}

\begin{document}

\section*{Integration}
\textbf{Integration | 163}

(m) \( \int x^3 \sin(x^2) \, dx \)

Let
\[
\begin{align*}
x^2 & = y \\
\frac{dy}{dx} & = 2x \quad \therefore \quad dy = 2x \, dx 
\end{align*}
\]

On substitution,
\[
\begin{align*}
\int x^3 \sin(x^2) \, dx & = \frac{1}{2} \int y^{\frac{1}{2}} \sin(y) \, dy \\
 & = \frac{1}{2} \left[ y^{\frac{1}{2}} \int \sin(y) \, dy - \int \left( \frac{d}{dy}\left( y^{\frac{1}{2}} \right) \int \sin(y) \, dy \right) \, dy \right]  \quad (Integration\, by\, parts)
\end{align*}
\]

Breaking down the next parts,

\[
\begin{align*}
 & = \frac{1}{2} \left[ \int y^{\frac{1}{2}} \sin(y) \, dy - \frac{1}{2} \int \cos(y) \, dy \right] \\
 & = \frac{1}{2} \left[ - y^{\frac{1}{2}} \cos(y) \, dy + \int \cos(y) \, dy \right] \\
 & = \frac{1}{2} \left[ - y^{\frac{1}{2}} \cos(y) + \sin(y) \right] + c \\
 & = \frac{1}{2} x^2 \left[ - \cos(x^2) + \sin(x^2) \right] + c
\end{align*}
\]
\intertext{Next part of the substitution,}
(n) \( \int \frac{dx}{\sqrt{1-x^2}} \)

Let
\[
\begin{align*}
x & = \sin \theta \\
\frac{dx}{d\theta} & = \cos \theta 
\end{align*}
\]

On substitution,

\[
\begin{align*}
\int \frac{dx}{\sqrt{1-x^2}} & = \int \frac{\cos \theta \, d\theta}{\sqrt{1-\sin^2\theta}} = \int d\theta = \theta + c = \sin^{-1}x + c
\end{align*}
\]

\intertext{o) \( \int \frac{dx}{1-x^2} = \frac{1}{2} \left( \int \frac{1}{1 + x} + \int \frac{1}{1 - x} \right) \)}

Let

\[
\begin{align*}
\int \frac{dx}{1-x^2} &= \frac{1}{2} \left( \int \frac{dx}{1 + x} + \int \frac{dx}{1 - x} \right) \\
&= \frac{1}{2} [ \ln(1+x) - \ln(1-x) ] + c \\
&= \frac{1}{2} \ln \left( \frac{1+x}{1-x} \right) + c
\end{align*}
\]

The integral can be written in another form,

\[
\begin{align*}
\int \frac{dx}{1-x^2} &= \frac{1}{2} \left( - \int \frac{dx}{x-1} + \int \frac{dx}{x+1} \right) \\
&= \frac{1}{2} \ln \left( \frac{x+1}{x-1} \right) + c
\end{align*}
\]

The appropriateness of the form of integral depends upon the value of \( x \). First form is appropriate for \( x < 1 \) and the second for \( x > 1 \).

(p) \( \int \frac{dx}{\sqrt{4-x^2}} \)

Let

\[
\begin{align*}
x & = 2 \sin \theta \\
\frac{dx}{d\theta} & = 2 \cos \theta 
\end{align*}
\]

On substitution,

\[
\begin{align*}
\int \frac{dx}{\sqrt{4-x^2}} &= \int \frac{2 \cos \theta \, d\theta}{\sqrt{4 - 4 \sin^2 \theta}} \\
&= \int d\theta \\
&= \theta + c = \sin^{-1} \left( \frac{x}{2} \right) + c
\end{align*}
\]

\end{document}
